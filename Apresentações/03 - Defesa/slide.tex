
%%%%%%%%%%%%%%%%%%%%%%%%%%%%%%%%%%%%%%%%%%%%%%%%%%%%%%%%%%%%%%%%%%%%%%%%%%%%%%%

% UNIVERSIDADE FEDERAL DO PARANÁ (UFPR)
% SETOR DE CIÊNCIAS SOCIAIS APLICADAS
% PÓS-GRADUAÇÃO EM DESENVOLVIMENTO ECONÔMICO (PPGDE)
% DISCENTE: FELIPE DUPLAT LUZ

%%%%%%%%%%%%%%%%%%%%%%%%%%%%%%%%%%%%%%%%%%%%%%%%%%%%%%%%%%%%%%%%%%%%%%%%%%%%%%%

%%%%%% APRESENTAÇÃO DE SLIDES - PPGDE UFPR %%%%%%

% Classe:
\documentclass[10pt]{sintefbeamer}

% Pacotes:
\usepackage{multirow}     % quebrar célula na tabela.
\usepackage[style = abnt, % manter formato ABNT.
            giveninits,   % manter primeiros nomes abreviados.
            scbib         % manter em versalete.
]{biblatex}               % adicionar referências.
\usepackage{amsmath}
\usepackage{bbm}

% Referências:
\addbibresource{Referências.bib}

% Capa:
\title{\large Comércio internacional, desigualdade de renda e pobreza: uma análise integrada de equilíbrio geral e microssimulação para o Brasil}
\subtitle{\textsc{\textcolor{black}{Orientador: Vinícius de Almeida Vale}}}
\author{\textsc{Co-orientadora: Kênia Barreiro de Souza \\ Felipe Duplat Luz}}
\date{\textsc{26 de fevereiro de 2024}}
\titlebackground{Imagens/marca_UFPR.png}



%%%%%%%%%%%%%%%%%%%%%%%%%%%%%%%%%%%%%%%%%%%%%%%%%%%%%%%%%%%%%%%%%%%%%%%%%%%%%%%


% DOCUMENTO COMEÇA A PARTIR DAQUI:


%%%%%%%%%%%%%%%%%%%%%%%%%%%%%%%%%%%%%%%%%%%%%%%%%%%%%%%%%%%%%%%%%%%%%%%%%%%%%%%

\begin{document}
\maketitle

% Primeira seção:
\section{Introdução}

\subsection{Motivações do projeto}

\begin{frame}{Motivações do projeto}
	\begin{itemize}[<+->]
		\item Texto.
	\end{itemize}
\end{frame}


\subsection{\textit{Gap} na literatura}

\begin{frame}{\textit{Gap} na literatura}
	\begin{itemize}[<+->]
		\item 
	\end{itemize}
\end{frame}


\subsection{Objetivo e possíveis contribuições}

\begin{frame}{Objetivo}
	\begin{itemize}[<+->]
		\item 
	\end{itemize}
\end{frame}

\begin{frame}{Possíveis contribuições}
	\begin{itemize}
		\item 
	\end{itemize}
\end{frame}



% Segunda seção:
\section{Revisão de literatura}

\subsection{Os canais de transmissão}

\begin{frame}{Os canais de transmissão}
	\begin{itemize}
		\item
	\end{itemize}
\end{frame}


\subsection{As evidências empíricas}

\begin{frame}{Equilíbrio parcial}
	
	Foco dos estudos:
	
	\begin{itemize}
		\item 
	\end{itemize}
\end{frame}

\begin{frame}{Equilíbrio geral}
	
	Foco dos estudos:
	
	\begin{itemize}
		\item 
	\end{itemize}
\end{frame}



% Terceira seção:
\section{Metodologia e dados}

\subsection{O modelo de Equilíbro Geral Computável}

\begin{frame}{O modelo EGC}
	\begin{itemize}[<+->]
		\item 
	\end{itemize}
\end{frame}


\subsection{Modelo de microssimulação}

\begin{frame}{Metodologia}
	\begin{itemize}[<+->]
		\item 
	\end{itemize}
\end{frame}


\subsection{Desigualdade de renda e pobreza}

\begin{frame}{Metodologia}
	\begin{itemize}[<+->]
		\item 
	\end{itemize}
\end{frame}



% Quarta seção:
\section{Simulação e resultados}

\subsection{Simulação e mecanismos de tranmissão}

\begin{frame}{Simulação}
	\begin{itemize}[<+->]
		\item 
	\end{itemize}
\end{frame}


\subsection{Resultados do modelo ORANIG-BR}

\begin{frame}{Simulação}
	\begin{itemize}[<+->]
		\item 
	\end{itemize}
\end{frame}


\subsection{Resultados da microssimulação comportamental}

\begin{frame}{Simulação}
	\begin{itemize}[<+->]
		\item 
	\end{itemize}
\end{frame}



% Quinta seção:
\section{Considerações finais}

\begin{frame}{Considerações finais}
	\begin{itemize}
		\item 
	\end{itemize}
\end{frame}



% Referências:
\bibliographpage



% Página final:
\backmatter



\end{document}


