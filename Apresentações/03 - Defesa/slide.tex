
%%%%%%%%%%%%%%%%%%%%%%%%%%%%%%%%%%%%%%%%%%%%%%%%%%%%%%%%%%%%%%%%%%%%%%%%%%%%%%%

% UNIVERSIDADE FEDERAL DO PARANÁ (UFPR)
% SETOR DE CIÊNCIAS SOCIAIS APLICADAS
% PÓS-GRADUAÇÃO EM DESENVOLVIMENTO ECONÔMICO (PPGDE)
% DISCENTE: FELIPE DUPLAT LUZ

%%%%%%%%%%%%%%%%%%%%%%%%%%%%%%%%%%%%%%%%%%%%%%%%%%%%%%%%%%%%%%%%%%%%%%%%%%%%%%%

%%%%%% APRESENTAÇÃO DE SLIDES - PPGDE UFPR %%%%%%

% Classe:
\documentclass[10pt]{sintefbeamer}

% Pacotes:
\usepackage{multirow}             % quebrar célula na tabela.
\usepackage[style = abnt,         % manter formato ABNT.
            giveninits,           % manter primeiros nomes abreviados.
            scbib                 % manter em versalete.
]{biblatex}                       % adicionar referências.
\usepackage{amsmath}              % fontes matemáticas.
\usepackage{threeparttable}       %
\usepackage{multicol}             %
\usepackage{tablefootnote}        %
\usepackage[cal=esstix]{mathalfa} % fontes matemáticas.
\usepackage{bbm}                  %
\usepackage{changepage}           % 
\usepackage{makecell}             %
\usepackage{multirow}             %
\usepackage{academicons}          % 

% para equações matemáticas:
\DeclareMathOperator\Log{Log}

% Referências:
\addbibresource{Referências.bib}

% Capa:
\title{\large Comércio internacional, desigualdade de renda e pobreza: uma análise integrada de equilíbrio geral e microssimulação para o Brasil}
\subtitle{\textsc{\textcolor{black}{Orientador: Vinícius de Almeida Vale}}}
\author{\textsc{Co-orientadora: Kênia Barreiro de Souza \\ Felipe Duplat Luz}}
\date{\textsc{26 de fevereiro de 2024}}
\titlebackground{Imagens/marca_UFPR.png}



%%%%%%%%%%%%%%%%%%%%%%%%%%%%%%%%%%%%%%%%%%%%%%%%%%%%%%%%%%%%%%%%%%%%%%%%%%%%%%%


% DOCUMENTO COMEÇA A PARTIR DAQUI:


%%%%%%%%%%%%%%%%%%%%%%%%%%%%%%%%%%%%%%%%%%%%%%%%%%%%%%%%%%%%%%%%%%%%%%%%%%%%%%%

\begin{document}
\maketitle

% Primeira seção:
\section{Introdução}

\subsection{Motivações da dissertação}

\begin{frame}{Motivações da dissertação}
	\begin{itemize}[<+->]
		\item Há uma extensa literatura que estuda os canais de transmissão entre o comércio internacional e a desigualdade de renda e pobreza:
		
		\begin{itemize}
			\item crescente destaque da abertura comercial como um vetor para o crescimento econômico \cite{atkin22};
			
			\item crença que a abertura é capaz de gerar melhorias sobre a produtividade e renda com repercussões positivas nos indicadores de desigualdade e pobreza \cite{carneiro06}.
		\end{itemize}
		
		\item Lastro na teoria econômica:
		
		\begin{itemize}
			\item Modelo Heckscher-Ohlin;
			\item Teorema Stolper-Samuelson;
			\item Princípio da compensação.
		\end{itemize}
	\end{itemize}
\end{frame}

\begin{frame}{Motivações da dissertação}
	\begin{itemize}[<+->]
		\item Entretanto, as evidências empíricas apontam para distintos cenários \cite{winters04}.
		
		\item Para os países em desenvolvimento, em especial o Brasil, esse debate é ainda mais impreciso:
		
		\begin{itemize}
			\item economias vulneráveis a choques externos \cite{bannisterthugge01}
			
			\item possível elevação do grau de incerteza \cite{winters02}
		\end{itemize}
		
		\item Isso não indica, necessariamente, que os estudos sejam inconclusivos; mas sim a inexistência de uma única resposta.
	\end{itemize}
\end{frame}


\subsection{\textit{Gap} na literatura}

\begin{frame}{\textit{Gap} na literatura}
	\begin{itemize}[<+->]
		\item A grande maioria dos estudos focou em analisar o tema a partir das experiências históricas de abertura comercial -- utilizando modelos de equilíbrio parcial \cite{castilho12, bayar17}.
		
		\item Os estudos que utilizaram modelos de equilíbrio geral não focaram na questão estrutural \cite{borrazetal12, estrades12, campostimini22}.
	\end{itemize}
\end{frame}


\subsection{Objetivo e contribuições à literatura}

\begin{frame}{Objetivo}
	\begin{itemize}[<+->]
		\item Estimar os efeitos de uma maior abertura comercial sobre os índices de desigualdade de renda e pobreza no Brasil.
		
		\item utiliza-se um modelo nacional de equilíbrio geral integrado a uma abordagem de microssimulação contrafactual referentes ao ano de 2015.
	\end{itemize}
\end{frame}

\begin{frame}{Contribuições à literatura econômica}
	\begin{itemize}
		\item Até onde se tem conhecimento no presente momento, apenas \textcite{carneiro06, ferreira06} conduziram um estudo semelhante para o Brasil, entretanto, sem realizar o mesmo nível de desagregação das famílias por percentis de renda.
		
		\item Esta dissertação contribui para a literatura econômica ao incorporar os efeitos do comércio internacional sobre a estrutura de renda das diferentes classes de famílias brasileiras, tanto entre si quanto entre indivíduos da mesma família.
	\end{itemize}
\end{frame}



% Segunda seção:
%\section{Revisão de literatura}

%\subsection{Os canais de transmissão}

%\begin{frame}{Os canais de transmissão}
%	\begin{figure}
%		\caption{canais de transmissão entre comércio internacional e desigualdade de renda e pobreza}
%		\includegraphics[width=10.5cm]{Imagens/canais_transmissao.ai} \\
%		\footnotesize
%		Fonte: \textcite{bannisterthugge01, xu03, goldbergpavcnik04, banerjee04}.
%	\end{figure}
%\end{frame}


%\subsection{As evidências empíricas}

%\begin{frame}{Foco nos estudos:}
%	
%	Equilíbrio parcial:
%	
%	\begin{itemize}
%		\item \textbf{mercado de trabalho} \cite{borjas94, forbes01, galianisanguinetti03}.
%		
%		\item \textbf{experiências históricas} \cite{castilho12, borrazetal12, bayar17}.
%	\end{itemize}
%
%	Equilíbrio geral:
%
%	\begin{itemize}
%		\item \textbf{experiências históricas} \cite{porto03, estrades12}.
%		
%		\item \textbf{simulações} \cite{ferreira06,corong14}.
%	\end{itemize}
%\end{frame}



% Terceira seção:
\section{Metodologia}

\subsection{Modelo de Equilíbro Geral Computável}

\begin{frame}{O modelo EGC}
	\begin{itemize}[<+->]
		\item Utilização do modelo nacional de Equilíbrio Geral Computável (ORANIG-BR) adaptado para cumprir os objetivos propostos:
		\begin{itemize}
			\item desagregação das famílias em percentis de renda (POF 08-09);
			\item desagregação do fator trabalho por nível de qualificação (PNAD 2015).
		\end{itemize}

		\item Modelo de tradição australiana da classe Johansen, partindo da estrutura teórica do ORANI \cite{dixit80}.

		\item Pode-se entender o modelo EGC enquanto um sistema de equações que objetivam descrever a dinâmica de uma economia a partir dos pressupostos walrasianos de equilíbrio geral \cite{horridge00}.
	\end{itemize}
\end{frame}

{
\setbeamertemplate{background canvas}{}
\begin{frame}[plain]
	\begin{figure}
		\includegraphics[width=13cm]{Imagens/estrutura_orani.ai} \\
		\footnotesize
	\end{figure}
\end{frame}
}


\subsection{Modelo de microssimulação}

\begin{frame}{Modelo de microssimulação}
	\begin{itemize}[<+->]
		\item O modelo de microssimulação pode ser entendido como uma grande variedade de técnicas de modelagem por meio das quais o comportamento ou estado dos indivíduos são estimados ou determinados \cite{figari15}.
		
		\item Integração macro-micro como alternativa à limitação do pressuposto da Família Representativa \cite{colombo08}.
	\end{itemize}
\end{frame}

\begin{frame}{Modelo de microssimulação}
	\begin{itemize}[<+->]
		\item Forma funcional \cite{bourguignon05}:
		\begin{align}
			\Log \mathcal{w}_{mi}  &= \alpha_{g} + \beta_{g} \mathcal{x}_{mi} + \upsilon_{mi} \hspace{3cm} i = 1, \ldots, k_m \label{eq:renda} \\[0.5cm]
			IW_{mi}                &= \mathbbm{1} \left[\gamma_{g} + \delta_{g} z_{mi} + \mu_{mi} \right] \label{eq:ocm} \\[0.1cm]
			\text{Y}_{m}           &= \sum_{i = 1}^{k_{m}} \mathcal{w}_{mi} IW_{mi} + y_{0m} \label{eq:renda_familiar}
		\end{align}
	\end{itemize}
\end{frame}

\begin{frame}{Modelo de microssimulação}
	\begin{itemize}[<+->]
		\item Correção de Heckman \cite{heckman79}:
		\begin{align}
			\boldsymbol{\hat{Trab}}_{g(mi)} &= \hat{\gamma}_g + \mathbf{X}_{g(mi)} \ \boldsymbol{\hat{\beta}}_g \label{eq:step1} \\
	\boldsymbol{\Lambda}_{g(mi)}    &= \frac{\phi(x)}{1 - \Phi(x)} \label{eq:mills} \\
			\mathbf{\Log \mathcal{\hat{w}}}_{g(mi)} &= \hat{\alpha}_g + \mathbf{Z}_{g(mi)} \ \boldsymbol{\hat{\beta}}_g \label{eq:step2}
		\end{align}

		\item Modelo de Escolha Ocupacional:
		\begin{align}
			\boldsymbol{\hat{Trab}}_{g(mi)} &= \hat{\gamma}_g + \mathbf{X}_{g(mi)} \ \boldsymbol{\hat{\beta}}_g \tag{4} \label{eq:step1}
		\end{align}
	\end{itemize}
\end{frame}

\begin{frame}{Modelo de microssimulação}
	\begin{figure}
		\caption{Estrutura esquemática da integração \textit{top-down}}
		\includegraphics[width=12cm]{Imagens/integracao_topdown.ai} \\
		\footnotesize
		Fonte: elaboração própria (2024) a partir de \textcite{tiberti17}.
	\end{figure}
\end{frame}


\subsection{Desigualdade de renda e pobreza}

\begin{frame}{Desigualdade de renda e pobreza}
	\begin{itemize}
		\item Índice de Gini:
		
		\begin{align}
			G = \frac{\sum_{i=1}^{n}\sum_{j=1}^{n} | x_i - x_j |}{2n^{2}\mu} \label{eq:gini}
		\end{align}

		\item Índices \textit{Foster-Greer-Thorbecke}:
		
		\begin{align}
			\text{FGT}_\alpha = \frac{1}{N} \sum_{i=1}^{Q} \left( \frac{z - y_i}{z} \right)^{\alpha}
		\end{align}

	\end{itemize}
\end{frame}



% Quarta seção:
\section{Resultados}

\subsection{Resultados do modelo ORANIG-BR}

\begin{frame}{Resultados do modelo ORANIG-BR}
	\begin{table}[h]
		\centering
		\tiny
		\begin{threeparttable}
			\caption{Efeitos de curto-prazo da redução tarifária sobre as importações (var. \%)} \label{tab:import}
			\begin{tabular}{m{3cm} >{\centering\arraybackslash}m{3cm} >{\centering\arraybackslash}m{3cm} >{\centering\arraybackslash}m{3cm}}
				\hline
				\multirow{2}{*}{\textit{\textbf{Commodities}}} & \multirow{2}{*}{\textbf{Setores agregados}} & \multicolumn{2}{c}{\textbf{Importações}} \\ \cline{3-4} 
					  &               & \textbf{Volume} & \textbf{Preço} \\ \hline
				 C38  & Indústria     &  2,7413         & -2,6082 \\
				 C40  & Indústria     &  2,6257         & -2,4429 \\
				 C39  & Indústria     &  1,9852         & -1,6339 \\
				 C27  & Agroindústria &  1,4747         & -0,5882 \\
				 C42  & Indústria     &  1,4279         & -0,8667 \\
				 C6   & Agropecuária  &  1,2962         & -0,9169 \\
				 C34  & Agroindústria &  1,2562         & -1,1298 \\
				 C63  & Indústria     &  1,2292         & -1,1688 \\
				 C37  & Agroindústria &  1,2211         & -1,7473 \\ \hline
				 C90  & Serviços      & -0,1760         & 0       \\
				 C87  & Serviços      & -0,1648         & 0       \\
				 C92  & Serviços      & -0,1412         & 0       \\
				 C102 & Serviços      & -0,0953         & 0       \\
				 C96  & Comércio      & -0,0925         & 0       \\
				 C111 & Serviços      & -0,0891         & 0       \\
				 C112 & Serviços      & -0,0868         & 0       \\
				 C23  & Extrativa     & -0,0798         & 0       \\
				 C94  & Serviços      & -0,0618         & 0       \\ \hline
				\end{tabular}
			\end{threeparttable}
	\end{table}
\end{frame}

\begin{frame}{Resultados do modelo ORANIG-BR}
	\begin{table}[h]
		\centering
		\tiny
		\begin{threeparttable}
			\caption{Efeitos de curto-prazo da redução tarifária sobre o nível de atividade e emprego (var. \%)}\label{tab:ativ}
			\begin{tabular}{m{2cm} >{\centering\arraybackslash}m{2cm} >{\centering\arraybackslash}m{2cm} >{\centering\arraybackslash}m{2cm} >{\centering\arraybackslash}m{2cm}}
				\hline
				\multirow{2}{*}{\textbf{Setores}} & \multirow{2}{*}{\textbf{Setores agregados}} & \multicolumn{2}{c}{\textbf{Indicadores}} \\ \cline{3-4} &  & \textbf{Nível de atividade} & \textbf{Emprego} \\ \hline
				C15 & Indústria     & 0,1849  & 0,2483 \\
				C35 & Indústria     & 0,1148  & 0,1379 \\
				C44 & Serviços      & 0,0809  & 0,1079 \\
				C37 & Serviços      & 0,0802  & 0,1472 \\
				C07 & Extrativa     & 0,0743  & 0,1413 \\
				C65 & Serviços      & 0,0661  & 0,0661 \\
				C60 & Serviços      & 0,0636  & 0,0690 \\
				C23 & Indústria     & 0,0618  & 0,0891 \\
				C62 & Serviços      & 0,0531  & 0,1027 \\
				C12 & Agroindústria & 0,0489  & 0,1476 \\ \hline
				C13 & Indústria     & -0,6101 & -0,7916 \\
				C14 & Indústria     & -0,1955 & -0,2686 \\
				C25 & Indústria     & -0,1342 & -0,1652 \\
				C36 & Indústria     & -0,1041 & -0,1985 \\
				C29 & Indústria     & -0,0368 & -0,0543 \\
				C33 & Indústria     & -0,0279 & -0,0298 \\
				C26 & Indústria     & -0,0251 & -0,0339 \\
				C16 & Indústria     & -0,0193 & -0,0314 \\
				C21 & Indústria     & -0,0099 & -0,0215 \\
				C27 & Indústria     & -0,0085 & -0,0149 \\ \hline
				\end{tabular}
			\end{threeparttable}
	\end{table}
\end{frame}

\begin{frame}{Resultados do modelo ORANIG-BR}
	\begin{table}[h]
		\centering
		\scriptsize
		\begin{threeparttable}
			\caption{Efeitos macroeconômicos de curto-prazo da redução tarifária} \label{tab:result}
			\begin{tabular}{m{8cm}c}
				\hline
				\multirow{2}{*}{\textbf{Indicadores}} & \multirow{2}{*}{\textbf{Var. (\%)}} \\
				 &  \\ \hline
				\textbf{Preços} &  \\
				\hspace{0.2cm} Índice de preços do consumidor & -0,1156 \\
				\hspace{0.2cm} Índice de preços do investimento & -0,1999 \\
				\hspace{0.2cm} Índice de preços do governo & -0,1103 \\
				\hspace{0.2cm} Índice de preços das exportações & -0,0973 \\
				\hspace{0.2cm} Índice de preços das importações & -0,4623 \\
				\hspace{0.2cm} Índice de preços do PIB & -0,1416 \\
				\hspace{0.2cm} Termos de troca & -0,0973 \\
				\hspace{0.2cm} Custos dos fatores primários & -0,0617 \\
				\hspace{0.2cm} Salário nominal & -0,1156 \\
				\hspace{0.2cm} Desvalorização real & 0,1418 \\ \hline
				\textbf{Volume} &  \\
				\hspace{0.2cm} Consumo real das famílias & 0,0555 \\
				\hspace{0.2cm} Volume exportado & 0,1009 \\
				\hspace{0.2cm} Volume importado & 0,1909 \\
				\hspace{0.2cm} PIB real & 0,0256 \\
				\hspace{0.2cm} Emprego real & 0,0358 \\ \hline
			\end{tabular}
		\end{threeparttable}
	\end{table}
\end{frame}


\subsection{Resultados da microssimulação comportamental}

\begin{frame}{Resultados da microssimulação comportamental}
	\begin{table}[h]
		\centering
		\scriptsize
		\begin{threeparttable}
			\caption{Microssimulação dos efeitos da redução tarifária sobre desigualdade de renda e pobreza por qualificação} \label{tab:result_microssimulacao}
			\begin{tabular}{m{5cm} >{\centering\arraybackslash}m{2cm} >{\centering\arraybackslash}m{2cm}}
				\hline
				\multirow{2}{*}{}                     & \multirow{2}{*}{\textbf{Simulado}} & \multirow{2}{*}{\textbf{Variação (\%)}} \\
													  &                           &                                \\ \hline
				\textbf{Pobreza$^{\dag}$}             &                           &                                \\
				\hspace{0.2cm} $\text{FGT}_0$         & 31,39                     & 0,024                          \\
				\hspace{0.2cm} $\text{FGT}_1$         & 14,34                     & 0,013                          \\
				\hspace{0.2cm} $\text{FGT}_2$         & 9,02                      & 0,010                          \\ \hline
				\textbf{Extrema pobreza$^{\ddagger}$} &                           &                                \\
				\hspace{0.2cm} $\text{FGT}_0$         & 8,64                      & --                             \\
				\hspace{0.2cm} $\text{FGT}_1$         & 4,30                      & 0,004                          \\
				\hspace{0.2cm} $\text{FGT}_2$         & 2,87                      & 0,003                          \\ \hline
				\textbf{Desigualdade de renda}        &                           &                                \\
				\hspace{0.2cm} Gini                   & 0,521                     & -0,007                         \\ \hline
				\end{tabular}
		\begin{tablenotes}
			\footnotesize
			\item \textit{Nota:}
			\item \hspace{0.2cm} $^{\dag}$     Indivíduos com renda familiar per capita abaixo de R\$367,02.
			\item \hspace{0.2cm} $^{\ddagger}$ Indivíduos com renda familiar per capita abaixo de R\$126,79.
		\end{tablenotes}
		\end{threeparttable}
	\end{table}
\end{frame}



% Quinta seção:
\section{Considerações finais}

\begin{frame}{Considerações finais}
	\begin{itemize}[<+->]
		\item Este trabalho tem como objetivo estimar os efeitos do comércio internacional sobre a distribuição de renda e pobreza no Brasil.

		\item \textbf{Resultado setorial}: ganhos para os setores da Agroindústria e parte da Indústria, mais voltada para o setor de calçados; perdas para boa parte da Indústria, especialmente os setores têxteis.
		
		\item \textbf{Resultado macroeconômico:} ganhos superaram perdas, uma vez que houve aumento do PIB real, emprego agregado e consumo das famílias.
		
		\item \textbf{Microssimulação:} variações bastante modestas nos índices de Gini e FGT . Aumento nos indicadores de pobreza, havendo sua maior variação na proporção de pobres; redução do indicador de distribuição de renda.
		
		\item \textbf{Possível explicação:} barreiras tarifárias já não seriam altas o suficiente para sua redução ser significativa. 
	\end{itemize}
\end{frame}

\begin{frame}{Considerações finais}
	\begin{itemize}
		\item Futuros trabalhos:
		\begin{itemize}[<+->]
			\item buscar novas desagregações do modelo ORANIG-BR;
			\item utilizar outras simulações que levem em conta a heterogeneidade da pauta exportadora;
			\item avançar na especificação do modelo ocupacional.
		\end{itemize}
	\end{itemize}
\end{frame}



% Referências:
\bibliographpage



% Página final:
\backmatter



\end{document}


