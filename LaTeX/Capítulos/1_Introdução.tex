
% ----------------------------------------------------------
% CAPÍTULO 01 - INTRODUÇÃO
% ----------------------------------------------------------

\chapter{Introdução} \label{cha:introdução}

Há uma extensa literatura que busca analisar o canal de transmissão entre o comércio internacional e a desigualdade de renda e pobreza \cite{ferreira06, castilho12, bayar17, anderson20}. Esse debate é motivado, por um lado, pelo crescente destaque da abertura comercial como um vetor para o crescimento econômico \cite{sala07} e, por outro lado, pela crença que essa abertura é capaz de gerar melhorias sobre a produtividade e renda com repercussões positivas nos indicadores de desigualdade e pobreza \cite{carneiroarbache03}.

Os modelos teóricos de economia internacional, por sua vez, apontam que o comércio é capaz de influir nos preços relativos de um país, gerando fortes efeitos distributivos sobre a sua renda nacional. Desse modo, espera-se que haja grupos beneficiados e grupos prejudicados a partir de uma determinada abertura comercial. Entretanto, a teoria também aponta que esses ganhos serão grandes o suficiente para compensar as perdas ocasionadas, dado o ganho de produtividade e bem-estar gerados pela maior exposição ao comércio internacional. O modelo H-O e o Teorema SS são dois exemplos que ilustram essa dinâmica.

No modelo H-O\footnote{Considera-se o modelo 2x2x2: dois países, dois fatores produtivos e dois bens.}, a abertura comercial gera, como consequência, um aumento de eficiência tanto na produção quanto no consumo do país. A referida mudança nos preços relativos causa uma alteração na produção de ambos os bens em ambos os países. Estes se especializam\footnote{Diferente do modelo ricardiano, aqui não há, necessariamente, especialização completa.} na produção do bem intensivo no seu fator produtivo abundante, tornando por exportá-lo, ao passo em que importa o bem que é intensivo no fator de produção escasso \cite{heckscher49, ohlin67}. Essa mudança, de acordo com o modelo, promove a elevação do bem-estar social do país.

Entretanto, esse ganho de produtividade não é igualmente repartido pela sociedade. De acordo com o teorema SS, o aumento do preço relativo de um bem, via efeito magnificação, também eleva a remuneração relativa do seu fator produtivo, reduzindo, por conseguinte, a remuneração do outro fator \cite{stolper41}. Ou seja, o aumento da renda dos proprietários de um fator produtivo resulta diretamente na redução da renda dos proprietários do outro fator. O comércio internacional sempre gera vencedores e perdedores.

A conclusão do teorema SS não impede de afirmar que o comércio internacional pode ser benéfico para todos. Se os ganhos excedem as perdas no movimento de liberalização comercial, é possível redistribuir a renda de tal forma que todos os indivíduos tenham, pelo menos, tanto quanto já tinham antes da abertura.

A isso, a teoria econômica conceitua como \textit{princípio da compensação} \cite{irwin98}. É a escolha política e econômica geralmente aceita sobre como lidar com os custos de uma liberalização comercial, podendo assumir diversas formas, incluindo pagamentos diretos, seguro salarial, retreinamento profissional ou até ajuda na transição para um novo emprego \cite{kolben21}. É a política preferencial a ser seguida para maximizar o bem-estar a partir de uma abertura comercial.

Apesar do consenso de que


