
%---------------------------------------------------------------------
% ABSTRACT
%---------------------------------------------------------------------

% ENG:
\begin{resumo}[Abstract]
	\begin{otherlanguage*}{english}
		\SingleSpacing
		
		Although most theoretical models of international economics converge on the understanding that trade can be a positive factor for a country's economic development, with positive effects on indicators of income inequality and poverty, empirical evidence points to different scenarios . One of the possible explanations for these divergences is related to the productive structure of each country and period, as well as the relationship between the productive structure and income distribution. In this sense, aiming to better understand the issue, this dissertation aims to estimate the effects of greater trade openness on the distribution of family income and on poverty rates in Brazil. For this, a national computable general equilibrium model integrated with a microsimulation model is used. While the first model allows calculating aggregate and sectoral effects of reducing tariff barriers, the second model allows accessing results at the individual level, calculating the possible effects on poverty and inequality indicators. The results indicated that international trade has little influence on income inequality and poverty. The variations recorded were quite modest, especially when it comes to the effect on extreme poverty and income inequality, which was practically nil. Even so, there was a modest reduction in income inequality and a small increase in absolute and extreme poverty. A possible reason that explains this result is the fact that the tariff barriers computed in the model are no longer high enough for a new tariff reduction to be able to impose significant effects on the observed indicators.
		
		\noindent 
		\textbf{Keywords}: International trade. Wage inequality. Poverty. Computable General Equilibrium. Behavioral microsimulation. \\
		\textbf{JEL Classification}: F10, F14, I32.
	\end{otherlanguage*}
\end{resumo}


