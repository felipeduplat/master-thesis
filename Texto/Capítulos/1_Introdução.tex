
% ----------------------------------------------------------
% CAPÍTULO 01 - INTRODUÇÃO
% ----------------------------------------------------------

\chapter{Introdução} \label{cha:introducao}

Há uma extensa literatura que busca analisar o canal de transmissão entre comércio internacional e a desigualdade de renda e pobreza \cite{ferreira06, castilho12, bayar17, anderson20}. Esse debate é motivado, por um lado, pelo crescente destaque da abertura comercial como um vetor para o crescimento econômico \cite{atkin22} e, por outro lado, pela crença que essa abertura é capaz de gerar melhorias sobre a produtividade e renda com repercussões positivas nos indicadores de desigualdade e pobreza \cite{carneiro06}.

Os modelos teóricos de economia internacional apontam, por sua vez, que o comércio é capaz de influir nos preços relativos de um país, gerando fortes efeitos distributivos sobre a sua renda nacional. Portanto, espera-se que haja grupos beneficiados e grupos prejudicados a partir de uma determinada abertura comercial. Entretanto, a teoria também aponta que esses ganhos serão grandes o suficiente para compensar as perdas ocasionadas, dado o aumento de produtividade e bem-estar gerados pela maior exposição ao comércio internacional. O modelo H-O \cite{heckscher49, ohlin67} e o Teorema SS \cite{stolper41} são dois exemplos que ilustram essa dinâmica.

No modelo H-O\footnote{Considera-se o modelo 2x2x2: dois países, dois fatores produtivos e dois bens.}, a abertura comercial gera, como consequência, um aumento de eficiência tanto na produção quanto no consumo do país. A referida mudança nos preços relativos causa uma alteração na produção de ambos os bens em ambos os países. Estes se especializam\footnote{Diferente do modelo ricardiano, aqui não há, necessariamente, especialização completa.} na produção do bem intensivo no seu fator produtivo abundante, tornando por exportá-lo, ao passo em que importa o bem que é intensivo no fator de produção escasso \cite{heckscher49, ohlin67}. Essa mudança, de acordo com o modelo, promove a elevação do bem-estar social do país.

Entretanto, esse ganho de produtividade não é igualmente repartido pela sociedade. De acordo com o teorema SS, o aumento do preço relativo de um bem, via efeito magnificação, também eleva a remuneração relativa do seu fator produtivo, reduzindo, por conseguinte, a remuneração do outro fator \cite{stolper41}. Ou seja, o aumento da renda dos proprietários de um fator produtivo resulta diretamente na redução da renda dos proprietários do outro fator. O comércio internacional sempre gera vencedores e perdedores.

A conclusão do teorema SS não impede de afirmar que o comércio internacional pode ser benéfico para todos. Se os ganhos excedem as perdas no movimento de liberalização comercial, é possível redistribuir a renda de tal forma que todos os indivíduos tenham, pelo menos, tanto quanto já tinham antes da referida abertura.

A isso, a teoria econômica conceitua como \textit{princípio da compensação} \cite{irwin98}. É a escolha política e econômica geralmente aceita sobre como lidar com os custos de uma liberalização comercial, podendo assumir diversas formas, incluindo pagamentos diretos, seguro salarial, retreinamento profissional ou até ajuda na transição para um novo emprego \cite{kolben21}. Ou seja, é a política preferencial a ser seguida para maximizar o bem-estar a partir de determinada uma abertura comercial.

Apesar da teoria econômica convergir para a noção que o comércio internacional é um fator capaz de reduzir a desigualdade de renda e pobreza, as evidências empíricas apontam para distintos cenários, não havendo qualquer consenso na literatura econômica sobre seus efeitos \cite{winters04}. 

Para os países latino-americanos, em especial o Brasil, essa questão é ainda mais dúbia, já que uma economia em desenvolvimento mais integrada ao comércio internacional também pode estar mais vulnerável a choques externos, como mudanças abruptas nos termos de troca, que podem reduzir significativamente o crescimento do país \cite{bannisterthugge01}. Essa vulnerabilidade eleva o grau de incerteza, fazendo com que o país possa operar com níveis de pobreza acima do que uma economia menos integrada operaria, além de gerar uma perda da eficiência de políticas econômicas capazes de reduzir pobreza e desigualdade de renda \cite{winters02}.

É válido ressaltar que a falta de consenso na literatura não indica, necessariamente, que os estudos sejam inconclusivos, mas pode indicar a inexistência de uma resposta única para a questão. O modelo H-O, bem como a extensa maioria dos modelos teóricos de comércio internacional, desconsidera a estrutura produtiva dos países, bem como a composição da pauta exportadora e a distribuição funcional da renda em sua formulação. A forma que essa diversidade de fatores pode gerar distintos impactos em termos de desigualdade de renda e pobreza é uma questão pouco explorada na literatura e, possivelmente, a razão da referida ausência de consenso.

Por essa razão, a presente dissertação tem como objetivo estimar os efeitos de uma maior abertura comercial sobre a distribuição da renda familiar e sobre os índices de pobreza no Brasil. Para isso, utiliza-se um modelo nacional de equilíbrio geral para simular diferentes cenários de políticas de liberalização comercial integrado a uma abordagem de microssimulações contrafactuais a fim de captar as respostas comportamentais dos indivíduos.

Embora a literatura econômica tenha abordado extensivamente esse assunto, sob diversas óticas, ainda há profícuos \textit{gaps} para serem adereçados. A grande maioria dos estudos se limitou a abordar o tema a partir das experiências históricas de abertura comercial - sendo comumente utilizado modelos de equilíbrio parcial \cite{castilho12, bayar17} - ou a partir de estudos de caso, sem focar na questão estrutural \cite{borrazetal12, estrades12, campostimini22}.

O Brasil serve como um interessante caso de estudo por algumas razões. Primeiro, pelo recente histórico de abertura comercial, seguindo a tendência de diversos países em desenvolvimento que, nas últimas quatro décadas, implementaram uma série de políticas liberalizantes em larga escala, integrando-se ao sistema de comércio global \cite{pavcnik17}, embora o coeficiente de abertura comercial brasileiro seja um dos menores do mundo, ocupando o nono lugar no ranking de países mais fechados ao comércio internacional \cite{ourworldindata}. Segundo, o Brasil ainda é um país com elevados índices de desigualdade de renda e pobreza, apesar de ter havido uma queda acentuada observada desde o início da década de 2000 \cite{ocde15}.

Até onde se tem conhecimento no presente momento, apenas \textcite{carneiro06, ferreira06} conduziram um estudo semelhante para o Brasil, entretanto, sem realizar o mesmo nível de desagregação das famílias e dos fatores produtivos do modelo de equilíbrio geral. Esta dissertação contribui para a literatura econômica ao incorporar os efeitos do comércio internacional sobre a estrutura de renda das diferentes classes de famílias brasileiras, tanto entre si quanto entre indivíduos da mesma família. 

A dissertação está organizada da seguinte forma: após esta Introdução, o segundo capítulo apresenta as mais recentes evidências da literatura sobre a interação entre comércio internacional e indicadores de desigualdade de renda e pobreza, relacionando-a com a evolução da estrutura produtiva brasileira; o terceiro capítulo detalha estratégia empírica empregada; o quarto capítulo expõe os resultados obtidos; e, por fim, o último capítulo apresenta as considerações finais.


