
% ----------------------------------------------------------
% CAPÍTULO 01 - INTRODUÇÃO
% ----------------------------------------------------------

% limpar customização do cabeçalho:
\pagestyle{fancy}
\fancyhf{}
\fancyhead[R]{\small\thepage}
\renewcommand{\headrulewidth}{0pt}

\chapter{Introdução} \label{cha:introducao}

Há uma extensa literatura que busca analisar o canal de transmissão entre comércio internacional e a desigualdade de renda e pobreza \cite{ferreira06, castilho12, bayar17, anderson20}. Esse debate é motivado, por um lado, pelo crescente destaque da abertura comercial como um vetor para o crescimento econômico \cite{atkin22} e, por outro lado, pela crença que essa abertura é capaz de gerar melhorias sobre a produtividade e renda com repercussões positivas nos indicadores de desigualdade de renda e pobreza \cite{carneiro06}.

A maioria dos modelos teóricos de economia internacional apontam, por sua vez, que o comércio é capaz de influir nos preços relativos de um país, gerando fortes efeitos distributivos sobre a sua renda nacional. Com isso, espera-se que haja grupos beneficiados e grupos prejudicados a partir de uma determinada abertura comercial. Entretanto, esses modelos também apontam que os ganhos serão grandes o suficiente para compensar as perdas ocasionadas, dado o aumento de produtividade e bem-estar gerados pela maior exposição ao comércio internacional. O modelo H-O \cite{heckscher49, ohlin67} e o Teorema SS, derivado do modelo anterior, \cite{stolper41} são dois exemplos que ilustram essa dinâmica.

No modelo H-O, a abertura comercial promove mudanças nos preços relativos. Os bens que são intensivos no uso do fator produtivo abundante no país terão seus preços relativos aumentados, pois a demanda por esses bens aumenta no mercado internacional; ao passo que os preços relativos dos bens que são intensivos no uso do fator de produção escasso tendem a diminuir. O resultado é a especialização do país no bem que usa intensivamente seu fator produtivo abundante, tornando por exportá-lo. O bem que usa mais intensivamente seu fator escasso é importado \cite{heckscher49, ohlin67}. Essa mudança de preços relativos, de acordo com o modelo, promove o aumento da eficiência tanto na produção quanto no consumo dos países, elevando seu nível de bem-estar.

Entretanto, esse ganho de produtividade não é igualmente repartido na sociedade. De acordo com o teorema SS, o aumento do preço relativo de um bem, via efeito magnificação\footnote{Qualquer mudança nos preços dos produtos gera um efeito ainda maior no preço dos fatores produtivos \cite{jones65}}, também eleva a remuneração relativa do seu fator produtivo, reduzindo, por conseguinte, a remuneração do outro fator \cite{stolper41}. Ou seja, o aumento da renda dos proprietários de um fator produtivo resulta diretamente na redução da renda dos proprietários do outro fator. Essa conclusão nos permite afirmar que o comércio internacional gera vencedores e perdedores.

Contudo, isso não significa que as perdas não possam ser compensadas. Se os ganhos excedem as perdas no movimento de liberalização comercial, é possível redistribuir os ganhos de tal forma que todos os indivíduos tenham, pelo menos, tanto quanto já tinham antes da referida liberalização. A isso, a teoria econômica conceitua como \textit{princípio da compensação} \cite{irwin98}, sendo entendida enquanto a escolha política e econômica sobre como lidar com os custos de uma liberalização comercial. Esta pode assumir diversas formas, incluindo pagamentos diretos, seguro salarial, retreinamento profissional ou até ajuda na transição para um novo emprego \cite{kolben21}. Ou seja, é a política preferencial a ser seguida para maximizar o bem-estar dada uma determinada uma abertura comercial.

Desse modo, pode-se afirmar que a teoria econômica converge para a compreensão de que o comércio internacional, mesmo ocasionando perdedores, é capaz de gerar efeitos suficientemente positivos para que, uma vez redistribuídos, coloquem todos os indivíduos numa situação, pelo menos, tão boa quanta era na situação de autarquia com repercussões positivas sobre os indicadores de desigualdade de renda e pobreza. Entretanto, as evidências empíricas apontam para distintos cenários sem encontrar a mesma convergência \cite{winters04}.

Para os países latino-americanos, em especial o Brasil, essa questão é ainda mais dúbia, já que uma economia em desenvolvimento mais integrada ao comércio internacional também pode estar mais vulnerável a choques externos, como mudanças abruptas nos termos de troca, que podem reduzir significativamente o crescimento do país \cite{bannisterthugge01}. Essa vulnerabilidade eleva o grau de incerteza, fazendo com que o país possa operar com níveis de pobreza acima do que uma economia menos integrada operaria, além de gerar uma perda da eficiência de políticas econômicas capazes de reduzir pobreza e desigualdade de renda \cite{winters02}.

É válido ressaltar que a falta de consenso na literatura não indica, necessariamente, que os estudos sejam inconclusivos, mas pode indicar a inexistência de uma resposta única para a questão. O modelo H-O, bem como a extensa maioria dos modelos teóricos de comércio internacional, desconsidera a estrutura produtiva dos países, bem como a composição da pauta exportadora e a distribuição funcional da renda em sua formulação. A forma que essa diversidade de fatores pode gerar distintos impactos em termos de desigualdade de renda e pobreza é uma questão pouco explorada na literatura e, possivelmente, a razão da referida ausência de consenso.

Embora a literatura econômica tenha abordado extensivamente esse assunto, sob diversas óticas, ainda há profícuos \textit{gaps} para serem adereçados. A grande maioria dos estudos se limitou a abordar o tema a partir das experiências históricas de abertura comercial - sendo comumente utilizados modelos de equilíbrio parcial \cite{castilho12, bayar17} - ou a partir de estudos de caso, sem focar na questão estrutural \cite{borrazetal12, estrades12, campostimini22}. Desse modo, pouco se debateu na literatura sobre a influência do padrão de comércio, e o perfil da pauta exportadora, bem como do padrão de consumo e renda das famílias, sobre a desigualdade de renda e pobreza, evidenciando os canais de transmissão que podem influenciar esses indicadores. 

Por essa razão, a presente dissertação tem como objetivo estimar os efeitos de uma maior abertura comercial sobre a distribuição da renda familiar e sobre os índices de desigualdade de renda e pobreza no Brasil. Para isso, utiliza-se um modelo nacional de equilíbrio geral integrado a uma abordagem de microssimulação contrafactual\footnote{A integração acaba por corrigir as limitações de ambos modelos. Essa discussão é desenvolvida no Capítulo~\ref{cha:metodologia}.} referentes ao ano de 2015. O primeiro cumpre o papel de calcular os efeitos macroeconômicos e setoriais de uma redução tarifária na magnitude de 10\%, enquanto que, através do segundo, é possível estimar esses efeitos a nível individual, calculando o impacto sobre a desigualdade de renda e pobreza.

Existe uma grande vantagem em trabalhar com modelos de equilíbrio geral para estudos sobre desigualdade de renda e pobreza em comparação a modelos de equilíbrio parcial. Sua estrutura é capaz de captar mais eficientemente os efeitos de reformas comerciais, especialmente sobre salários e emprego -- determinantes do impacto geral de aberturas comerciais \cite{naranpanawa11}. Além disso, não há problemas comuns de abordagens econométricas como, por exemplo, viés de seleção, heterogenidade e dificuldade em separar os efeitos de múltiplas reformas introduzidas simultaneamente \cite{anderson20}.

O Brasil serve como um interessante estudo de caso por duas razões. Primeiro, pelo recente histórico de abertura comercial, seguindo a tendência de diversos países em desenvolvimento que, nas últimas quatro décadas, implementaram uma série de políticas liberalizantes em larga escala, integrando-se ao sistema de comércio global \cite{pavcnik17}, embora seu coeficiente de abertura comercial seja um dos menores do mundo, ocupando o nono lugar no ranking de países mais fechados ao comércio internacional\footnote{Dados do \textcite{ourworldindata}.}. Segundo, o Brasil ainda é um país com elevados índices de desigualdade de renda e pobreza, apesar de ter havido uma queda acentuada desde o início da década de 2000 \cite{ocde15}.

Até onde se tem conhecimento no presente momento, apenas \textcite{carneiro06, ferreira06} conduziram um estudo semelhante para o Brasil, entretanto, sem realizar o mesmo nível de desagregação das famílias por percentis de renda. Esta dissertação contribui para a literatura econômica ao incorporar os efeitos do comércio internacional sobre a estrutura de renda das diferentes classes de famílias brasileiras, tanto entre si quanto entre indivíduos da mesma família.

A estrutura desta dissertação segue a seguinte ordem: após a Introdução, o segundo capítulo aborda as evidências da literatura acerca da interação entre o comércio internacional e os indicadores de desigualdade de renda e pobreza. O terceiro capítulo detalha a estratégia empírica aqui adotada. O quarto capítulo detalha as simulações e os resultados obtidos. Por fim, o quinto e último capítulo apresenta as considerações finais.


