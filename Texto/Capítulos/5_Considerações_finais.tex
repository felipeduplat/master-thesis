
% ----------------------------------------------------------
% CAPÍTULO 05 - CONSIDERAÇÕES FINAIS
% ----------------------------------------------------------

\chapter{Considerações finais}

Este trabalho tem como objetivo estimar os efeitos do comércio internacional sobre a distribuição da renda familiar e sobre os indicadores de pobreza no Brasil. Para isso, optou-se por utilizar um modelo nacional de equilíbrio geral computável integrado uma abordagem de microssimulações comportamentais que permite, em conjunto, avaliar os resultados tanto a nível macroeconômico quanto microeconômico.

Como simulação, foi proposto uma redução tarifária no montante de 10\% sobre todas as \textit{commodities} da economia para observar os efeitos de curto-prazo de uma política orientada para a liberalização comercial. Os resultados macroeconômicos apontaram para ganhos nos setores da Agroindústria e parte da Indústria, mais voltada para o setor de tecidos e calçados -- pouco intensivos em trabalho -- ao passo em que se registrou perdas para boa parte da Indústria, especialmente os setores têxteis. Também se observou que os ganhos foram grandes o suficiente para superar as perdas, uma vez que houve aumento do PIB real, emprego agregado e consumo das famílias.

Entretanto, essa maior exposição ao comércio internacional não promoveu melhorias e tampouco deteriorações nos indicadores de desigualdade de renda. O mesmo pode ser dito para os indicadores de pobreza no Brasil. As variações observadas nos índices de Gini e FGT foram bastante modestas, sobretudo quando se trata de extrema pobreza e desigualdade de renda, no qual a influência do comércio internacional foi praticamente nula. Isso converge para os resultados de \textcite{carneiro06}: liberalizações comerciais são pouco eficazes para influenciar esses indicadores.

Mesmo assim, apesar do diminuto efeito, o modelo de microssimulação comportamental registrou aumento generalizado nos indicadores de pobreza absoluta e extrema para as três categorias de qualificação, havendo sua maior variação na proporção dos extremamente pobres entre os não qualificados e dos pobres entre os semi-qualificados e qualificados, além de haver uma expansão do \textit{gap} da pobreza para os semi-qualificados. Sobre a desigualdade de renda, a direção do efeito foi contrária: observou-se uma redução para os três grupos. A direção desses resultados converge com as evidências encontradas por \textcite{borrazetal12}.

Por fim, várias razões podem explicar esses resultados. Apesar de não ser possível testá-las neste trabalho, uma possível explicação para a diminuta influência do comércio internacional esteja ancorada no fato em que as barreiras tarifárias brasileiras já não sejam altas o suficiente para que uma redução gere resultados expressivos. Sobre os efeitos divergentes da desigualdade de renda e pobreza, uma possível explicação esteja na estrutura produtiva brasileira, no qual a redução tarifária prejudicou os setores comparativamente mais intensivos em trabalho, gerando, como consequência, elevação dos níveis de pobreza.


