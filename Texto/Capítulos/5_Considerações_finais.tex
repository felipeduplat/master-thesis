
% ----------------------------------------------------------
% CAPÍTULO 05 - CONSIDERAÇÕES FINAIS
% ----------------------------------------------------------

\chapter{Considerações finais}

Este trabalho tem como objetivo estimar os efeitos do comércio internacional sobre a distribuição da renda familiar e sobre os indicadores de pobreza no Brasil. Para isso, optou-se por utilizar um modelo nacional de equilíbrio geral computável de forma \textit{top-down} a uma abordagem de microssimulação paramétrica comportamentais que permite, em conjunto, avaliar os resultados tanto a nível macroeconômico quanto microeconômico.

Como simulação, foi proposta uma redução tarifária no montante de 10\% sobre todas as \textit{commodities} da economia para observar os efeitos de curto-prazo de uma política orientada para a liberalização comercial. Os resultados macroeconômicos apontaram para ganhos nos setores da Agroindústria e parte da Indústria, mais voltada para o setor de tecidos e calçados -- pouco intensivos em trabalho -- ao passo em que os resultados indicam perdas para boa parte da Indústria, especialmente os setores têxteis. A simulação também indica que os ganhos foram grandes o suficiente para superar as perdas, uma vez que houve aumento do PIB real, emprego agregado e consumo das famílias.

Entretanto, essa maior exposição ao comércio internacional não promoveu grandes melhorias ou deteriorações nos indicadores de desigualdade de renda. O mesmo pode ser dito para os indicadores de pobreza no Brasil. As variações observadas nos índices de Gini e FGT foram bastante modestas, sobretudo quando se trata de extrema pobreza e desigualdade de renda, no qual a influência do comércio internacional foi praticamente nula.

Mesmo assim, apesar do diminuto efeito, o modelo de microssimulação registrou aumento nos indicadores de pobreza, havendo sua maior variação na proporção de pobres, além de registrar uma expansão do \textit{gap}. Sobre a desigualdade de renda, a direção do efeito foi contrária: observou-se uma redução do índice de Gini.

Por fim, várias razões podem explicar esses resultados. Uma possível justificativa para a diminuta influência do comércio internacional esteja ancorada no fato em que as barreiras tarifárias brasileiras já não sejam altas o suficiente para que uma redução gere resultados expressivos. Sobre os efeitos divergentes da desigualdade de renda e pobreza, uma possível explicação esteja na estrutura produtiva brasileira, no qual a redução tarifária prejudicou os setores comparativamente mais intensivos em trabalho, gerando, como consequência, elevação dos níveis de pobreza.

A partir desses resultados, é possível traçar dois caminhos para futuros trabalhos nesta temática. Pelo ponto de vista empírico, pode-se testar novas simulações no Modelo EGC que levem em conta a heterogeneidade da pauta exportadora e a estrutura produtiva, além de poder testar se o tamanho das barreiras tarifárias é, de fato, um vetor que está influenciando os efeitos da microssimulação. Pelo ponto de vista metodológico, buscar novas desagregações no modelo de equilíbrio geral computável para o vetor das exportações e importações poderia refinar os resultados macroeconômicos e setoriais encontrados, bem como avançar na especificação da escolha ocupacional e utilizar novos indicadores que mensurem a desigualdade de renda, como, por exemplo, o índice de Theil.


