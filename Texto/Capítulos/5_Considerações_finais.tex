
% ----------------------------------------------------------
% CAPÍTULO 05 - CONSIDERAÇÕES FINAIS
% ----------------------------------------------------------

\chapter{Considerações finais}

Apesar da crença, corroborada pela teoria econômica, de que o comércio internacional pode engajar a economia de um país num ritmo de crescimento acelerado com \textit{spillovers} positivos sobre os indicadores de desigualdade de renda e pobreza \cite{carneiro06}, as evidências empíricas apontam para distintos cenários, não havendo qualquer consenso na literatura econômica sobre seus efeitos. Para os países em desenvolvimento, essa questão é ainda mais dúbia, já que uma economia em desenvolvimento mais integrada ao comércio internacional também pode estar mais vulnerável a choques externos, como mudanças abruptas nos termos de troca, que podem reduzir significativamente o crescimento do país \cite{bannisterthugge01}.

Entretanto, isso não necessariamente significa que não haja uma resposta conclusiva na literatura, mas sim pode indicar que não existe uma resposta única. É bastante plausível argumentar que o efeito composição da estrutura produtiva, da distribuição funcional da renda e do perfil da pauta exportadora influenciam significativamente na natureza do impacto de uma liberalização comercial sobre as variáveis de desigualdade de renda e pobreza. O modelo H-O, bem como a extensa maioria dos modelos teóricos de comércio internacional, desconsidera esses efeitos de equilíbrio geral. E a forma que essa diversidade de fatores pode gerar distintos impactos em termos de desigualdade de renda e pobreza é uma questão pouco explorada na literatura e, possivelmente, a razão da referida ausência de consenso.

Por essa razão, a presente dissertação tem como objetivo estimar os efeitos de uma maior abertura comercial sobre a distribuição da renda familiar e sobre os índices de pobreza no Brasil através do modelo nacional de equilíbrio geral para simular diferentes cenários de políticas de liberalização comercial integrado a uma abordagem de microssimulações contrafactuais para capturar as respostas comportamentais dos indivíduos. 

Propôs-se um corte tarifário de 25\% para todos os setores da economia como um exercício contrafactual a fim de avaliar o efeito de curto prazo do comércio internacional sobre a desigualdade de renda e pobreza no Brasil. Os resultados apontaram que o corte tarifário atuou como um choque positivo de produtividade para os setores com maior aderência ao comércio internacional, ao passo em que penalizou os setores voltados para o mercado doméstico. Como os ganhos setoriais não foram capazes de suprir as perdas, não houve ganho social líquido. A deterioração dos termos de troca, somado a uma relevante perda de receita tarifária, fizeram com que o cenário macroeconômico piorasse, havendo queda do PIB, emprego e salários.

É válido ressaltar que essa simulação não é o suficiente para poder realizar qualquer tipo de inferência ou associação entre o comércio internacional e os indicadores de desigualdade de renda e pobreza. Ainda é necessário sofisticar o método utilizado, desagregando as exportações pelos maiores parceiros comerciais do Brasil; realizar mais simulações que consigam mapear os efeitos tanto pelo lado de barreiras tarifárias e não-tarifárias, como também pelo lado dos acordos comerciais e efeito dos blocos regionais; e, por fim, avançar na integração do modelo de equilíbrio geral a uma abordagem de microssimulações contrafactuais para poder captar o efeito ao nível microeconômico.


