
% ----------------------------------------------------------
% CAPÍTULO 02 - REVISÃO DE LITERATURA
% ----------------------------------------------------------

\chapter{Revisão de literatura}\label{cha:revisao_de_literatura}

Este capítulo contextualiza as contribuições, tanto teóricas quanto empíricas, da literatura econômica sobre os efeitos do comércio internacional sobre a desigualdade de renda e pobreza. A primeira seção discute as contribuições teóricas, concentrando-se nos canais de transmissão que associam o comércio internacional à desigualdade e pobreza. Nele, discute-se a natureza desse vínculo e os comportamentos esperados. A segunda e última seção apresenta as evidências empíricas existentes sobre os referidos canais de transmissão e os distintos cenários apontados.



\section{Os canais de transmissão}\label{sec:canais_de_transmissao}

De início, é importante afirmar que estabelecer uma associação entre comércio internacional e indicadores de desigualdade de renda e pobreza é uma atividade desafiadora. A própria mensuração de desigualdade e pobreza é bastante complexa, sendo, por si só, tema exclusivo de diversos estudos \cite{neri06, soares09, hoffmann19}. Ademais, há outro grande desafio em desvincilhar os próprios canais de transmissão entre si, uma vez que são interdependentes e sujeitos a influência de outros tipos de políticas e eventos econômicos \cite{bannisterthugge01}.

Pode-se entender o comércio internacional enquanto um choque exógeno sobre os preços relativos de uma economia que, segundo \textcite{winters02}, gera fortes efeitos distributivos entre e dentro dos países. O comércio internacional pode afetar os preços relativos de uma economia de diversas formas: por variações cambiais, inovação, concorrência, entre outras. No modelo H-O, a abertura comercial promove a convergência dos preços relativos entre os dois países a partir da entrada do bem por um preço abaixo do praticado em autarquia.

Para o autor, o resultado final desse choque depende da estrutura de mercado e particularmente do setor de distribuição -- entendido enquanto uma categoria genérica para agrupar todos os canais de transmissão existentes em uma economia \cite{winters02}. Entretanto, frente a um processo de liberalização comercial, se a estrutura de mercado e setor de distribuição performam aquém do esperado, é possível que haja determinadas regiões do país e setores produtivos que fiquem isolados dos efeitos positivos desse processo \cite{bannisterthugge01}.

Intuitivamente, o primeiro canal de transmissão entre o comércio internacional e a desigualdade de renda e pobreza que se pode imaginar é o crescimento econômico, uma vez que um maior crescimento indica maiores oportunidades de emprego e de geração de renda, que afetam diretamente a pobreza e desigualdade\footnote{Há uma extensa literatura econômica acerca dos efeitos do crescimento pró-pobre e pró-rico (\textit{pro-poor and pro-rich growth}) sobre a desigualdade de renda. A respeito disso, pode-se citar \textcite{ravallion04, thorbecke22}.}. Entretanto, sua conexão com o comércio internacional é indefinida, sendo também tema de diversos estudos \cite{anderson16, dix17, gnangnon18}. Em razão disso, é descartado o crescimento econômico como possível canal de transmissão. 

A primeira subseção discute os canais de transmissão conhecidos na literatura entre o comércio internacional e a desigualdade de renda. Já a segunda subseção discute os canais de transmissão que afetam a pobreza. A Figura~\ref{fig:canais_de_transmissao} resume em tópicos a discussão realizada a seguir.

%\begin{quadro}[h]
%	\begin{threeparttable}
%		\centering
%		\caption{Canais de transmissão entre comércio internacional e a desigualdade de renda e pobreza}
%		\footnotesize
%		\label{quad:canais_de_transmissao}
%		\begin{tabular}{|| m{7.5cm} | m{7.5cm} ||}
%			\hline \hline
%			\multicolumn{1}{||c|}{\textbf{desigualdade de renda}} & \multicolumn{1}{c||}{\textbf{pobreza}} \\ \hline
%			\begin{itemize}
%				\item prêmio salarial por qualificação
%				\item prêmio salarial por setor
%				\item emprego informal
%				\item imperfeições do mercado de crédito
%				\item quantidade de bens não-negociáveis
%			\end{itemize} &
%			\vspace{0.2cm}
%			alteração no:
%			\begin{itemize}
%			\item preço e acesso dos produtos negociáveis
%			\item preço relativo dos fatores de produção
%			\item receitas e capacidade de gastos do governo
%			\item incentivos de investimentos e inovação
%			\item vulnerabilidade a choques externos
%			\end{itemize} \\ \hline \hline
%		\end{tabular}
%		\begin{tablenotes}
%			\scriptsize
%			\item Fonte: \textcite{bannisterthugge01, xu03, goldbergpavcnik04, banerjee04}.
%		\end{tablenotes}
%	\end{threeparttable}
%\end{quadro}

\begin{figure}[H]
	\centering
	\caption{Canais de transmissão entre comércio internacional e a desigualdade de renda e pobreza} \label{fig:canais_de_transmissao}
	\includegraphics[width=\textwidth]{Imagens/canais_transmissao.ai} \\
	\footnotesize
	Fonte: elaboração própria (2024) com base em \textcite{bannisterthugge01, xu03, goldbergpavcnik04, banerjee04}.
\end{figure}


\subsection{Desigualdade de renda} \label{subsec:desigualdade_de_renda}

Analisando a desigualdade de renda, o primeiro ponto da Figura~\ref{fig:canais_de_transmissao} se refere ao prêmio salarial por qualificação. Este pode ser entendido enquanto a diferenciação nos salários devido ao nível de qualificação, habilidades ou educação de um trabalhador. Políticas comerciais podem afetar o prêmio salarial a partir de: 1- um aumento nos retornos de determinadas ocupações associadas a um nível de escolaridade maior; 2- um deslocamento da produção de bens intermediários intensivos em qualificação dos países desenvolvidos para os em desenvolvimento; 3- uma mudança tecnológica com viés para maior qualificação; e 4- uma alteração da composição nos produtos produzidos dentro das indústrias em prol dos intensivos em trabalho qualificado \cite{goldbergpavcnik04}.

De acordo com \textcite{goldbergpavcnik04}, seu comportamento é positivamente relacionado com os indicadores de desigualdade. Ou seja, uma redução no prêmio salarial por qualificação diminui a desigualdade de renda. Isso ocorre porque o prêmio salarial afeta diretamente o \textit{gap} entre dois ou vários salários, agravando a desigualdade de renda. Esse comportamento é convergente com os modelos teóricos de comércio internacional, em especial com o modelo H-O -- considerando dois países, desenvolvido e em desenvolvimento, cujo primeiro é abundante em trabalho qualificado e o segundo, em não-qualificado -- e o teorema SS. Este garante que haveria uma redução salarial dos trabalhadores qualificados, aumentando o dos não-qualificados quando ambos se engajarem em comércio.

Em segundo, a Figura~\ref{fig:canais_de_transmissao} cita o prêmio salarial por setor como outro possível canal de transmissão. Segundo \textcite{goldbergpavcnik04}, este representa a variação salarial que pode existir entre trabalhadores empregados em setores diferentes da atividade econômica. Há duas formas que políticas comerciais podem afetar o prêmio por setor. Primeiro, rigidez na mobilidade do fator trabalho. Modelos de comércio de curto-prazo com a referida rigidez apontam que reduções tarifárias diminuem o prêmio salarial por setor \cite{heckman00}. Isso pode ocorrer em mercados de concorrência perfeita ou imperfeita \cite{harrison03}. Segundo, mudanças de produtividade intersetorial. Supondo que o comércio internacional possa aumentar a produtividade de um país, esta pode ser revertida em aumento no prêmio por setor desde que possa aumentar o salário dos trabalhadores \cite{hay01}.

Conforme \textcite{goldbergpavcnik04}, seu comportamento é negativamente relacionado com a desigualdade de renda. Ou seja, uma redução do prêmio por setor eleva o \textit{gap} salarial. Isso pode acontecer se uma abertura comercial reduz o prêmio. Isto pode aumentar a desigualdade caso os setores mais afetados pela abertura sejam intensivos em trabalho não qualificado \cite{pavcnik04}.

Em terceiro, a Figura~\ref{fig:canais_de_transmissao} elenca o emprego informal. A informalidade também é um canal de transmissão capaz de influir sobre os indicadores de desigualdade de renda. Isso acontece porque, segundo \textcite{goldbergpavcnik04}, uma abertura comercial pode ser capaz de expandir o tamanho do mercado informal de um país \cite{goldberg03}. Uma vez que os trabalhadores informais estão associados a menores salários e qualidade da infraestrutura do trabalho \cite{bargain14}, é possível afirmar que seu comportamento é positivamente relacionado com os indicadores de desigualdade de renda. Ou seja, uma abertura comercial pode aumentar a informalidade, causando o aumento da desigualdade de renda.

Em quarto, cita-se o mercado de crédito. Ele pode ser influenciado pelo comércio internacional, sendo, também, um canal de transmissão sobre a desigualdade. De acordo com \textcite{goldbergpavcnik04}, sua eficiência é negativamente relacionada com os indicadores de desigualdade de renda.

\textcite{banerjee04} desenvolvem um modelo dinâmico de comércio internacional com fatores específicos que incorpora imperfeição na mobilidade do capital e trabalho -- só há a transição de trabalho para capital se o indivíduo tiver uma quantidade mínima de capital humano. Frente a uma abertura comercial, observa-se o aumento da taxa de juros nos países com mercado de crédito mais robusto e consequente aumento da produtividade média, ao passo que os países com mercado de crédito pouco eficiente sofrem uma redução em sua taxa de juros, fazendo com que o capital permaneça nos setores menos produtivos - uma vez que aumenta a probabilidade de inadimplência. O efeito sobre a distribuição de renda é um \textit{trade-off} para os países em desenvolvimento: maior crescimento econômico no longo prazo às custas de alta volatilidade no curto-prazo.

O último canal de transmissão listado é o que a literatura econômica chama de bens não-negociáveis. Entende-se por isso todos os bens cuja características dificultam ou impedem de serem exportados ou importados. \textcite{xu03}, utilizando um um modelo H-O com dois países (norte e sul), dois fatores (trabalho qualificado e não-qualificado) e quatro bens (1, 2, 3 e 4 numa escala crescente de complexidade), argumenta que os efeitos da abertura comercial sobre a desigualdade de renda nos países em desenvolvimento são em forma de U. Para níveis de proteção comercial inicialmente elevados, a abertura reduz a desigualdade, enquanto para níveis de proteção mais baixos, a abertura aumenta a desigualdade.

Isso ocorre por causa da gama de bens não-negociáveis: a proteção comercial transforma alguns bens negociáveis em não-negociáveis via distorção nas vantagens comparativas. Se esses bens são intensivos em trabalho qualificado, uma progressiva abertura comercial, via redução tarifária, aumenta os incentivos para importar. A consequência disso é a redução do salário do trabalho qualificado no país, reduzindo o \textit{gap} salarial - como visto no Teorema SS. Entretanto, a redução tarifária também provoca a piora dos termos de troca, o que torna por incentivar as exportações do país - o que elevaria o salário do trabalho qualificado, aumentando a desigualdade. Uma elevada abertura comercial faria com que o "efeito exportação" superasse o "efeito importação", criando, por conseguinte, a forma de U para os países em desenvolvimento \cite{goldbergpavcnik04}.


\subsection{Pobreza} \label{subsec:pobreza}

A segunda coluna da Figura~\ref{fig:canais_de_transmissao} lista os canais de transmissão entre o comércio internacional e os indicadores de pobreza. Os indivíduos mais pobres podem ser beneficiados por uma abertura comercial, a partir da redução do preço dos bens importados, como alimentos básicos e farmacêuticos, gerando o aumento da renda real, além de aumentar o acesso a novos bens de consumo \cite{bannisterthugge01}. Como geralmente os indivíduos mais pobres são produtores líquidos de bens voltados para a exportação (como na Agricultura, por exemplo), o aumento da demanda por exportações pode elevar o preço dos produtos e, por conseguinte, estimular o aumento da produção, emprego e renda do setor \cite{bannisterthugge01}.

A alteração no preço relativo dos fatores de produção, como apresenta a Figura~\ref{fig:canais_de_transmissao}, também é um canal de transmissão sobre os indicadores de pobreza, uma vez que afeta diretamente a remuneração do trabalho qualificado e não-qualificado. Como já visto no modelo H-O, o comércio internacional pode beneficiar os mais pobres caso a abertura se reverta numa elevação da produtividade marginal do trabalho não-qualificado \cite{bannisterthugge01}.

A Figura~\ref{fig:canais_de_transmissao} cita as receitas e capacidade de gastos do governo como outro canal de transmissão para o comércio internacional. Isso ocorre porque políticas de liberalização comercial, feitas por meio de cortes tarifários, afetam diretamente as receitas tarifárias, o que pode prejudicar a capacidade de gastos do governo.

De acordo com \textcite{bannisterthugge01}, o efeito sobre a pobreza se baseia no argumento que essa redução na receita tarifária teria que ser compensada pelo governo por meio de cortes em programas sociais ou criação de novos impostos, afetando desproporcionalmente os mais pobres. Entretanto, o resultado é ambíguo, sendo dependente do valor inicial das tarifas. Caso sejam altas, o corte tarifário eleva o fluxo comercial de tal maneira a compensar as perdas tributárias, além de diminuir os incentivos ao contrabando e corrupção. Caso sejam suficientemente baixas, não é possível compensar a perda da receita tarifária.

Investimentos e inovação, citado na Figura~\ref{fig:canais_de_transmissao}, também são um possível canal de transmissão. A literatura econômica associa reformas comerciais a maiores fluxos de investimento externo com \textit{spillovers} sobre tecnologias, novas práticas de negócios e outros efeitos sobre as empresas nacionais que aumentam o nível geral de produtividade\footnote{Reformas comerciais também resultam na melhoria da formação de capital humano, que gera um significante efeito sobre inovação \cite{bannisterthugge01}.} \cite{bannisterthugge01}. Entretanto, é possível que esse tipo de aumento de produtividade engendre um crescimento econômico que beneficie desproporcionalmente os mais ricos, o que agravaria os indicadores de pobreza \cite{lundberg03}.

O último canal de transmissão listado na Figura~\ref{fig:canais_de_transmissao} se refere a vulnerabilidade a choques externos. Aberturas comerciais podem facilitar a diversificação da pauta exportadora de um país, tornando-o menos dependente de um único mercado ou bem; como pode, também, tornar a economia mais vulnerável a choques externos. O canal de transmissão com indicadores de pobreza, nesse caso, se dá quando esse choque recai sobre setores intensivos em trabalho não-qualificado, como Agricultura ou qualquer setor muito caracterizado por trabalho informal, afetando desproporcionalmente os mais pobres \cite{bannisterthugge01}. Desse modo, a abertura comercial gera efeitos ambíguos: há evidências de elevação do crescimento econômico, entretanto, com altos níveis de volatilidade macroeconômica - o que, mais uma vez, afeta desproporcionalmente os mais pobres.



\section{As evidências empíricas} \label{sec:evidencias_empiricas}

Conforme discutido no Capítulo~\ref{cha:introducao}, apesar da teoria econômica convergir para a noção que o comércio internacional é um fator capaz de reduzir a desigualdade de renda e pobreza, as evidências empíricas apontam para distintos cenários. \textcite{winters02} elenca cinco razões para o \textit{mismatch} entre a teoria e as evidências empíricas: 1- distribuição funcional da renda; 2- dimensionalidade dos modelos; 3- pressuposto de mobilidade dos fatores; 4- equilíbrio diverso; e 5- comportamento do preço dos bens não negociáveis para \textit{market-clearing}.

Para analisar esses distintos cenários, optou-se por apresentar a contribuição empírica a partir do aspecto metodológico. A primeira subseção discute as contribuições a partir de modelos de equilíbrio parcial e a segunda, a partir de modelos de equilíbrio geral. Essa divisão é adotada para proporcionar uma compreensão mais aprofundada de como essa escolha pode afetar, de algum modo, os resultados sobre desigualdade de renda e pobreza, como discutido em \textcite{anderson20}.


\subsection{Equilíbrio parcial} \label{subsec:eq_parcial}

Um dos mais tradicionais canais de transmissão, no equilíbrio parcial, é o mercado de trabalho. Nesse caminho, \textcite{borjas94} buscam verificar se a tendência de exposição dos setores altamente concentrados ao comércio internacional durante 1963 a 1988 nos Estados Unidos podem ser responsáveis por grande parte das tendências na desigualdade salarial. Usando um modelo de séries temporais, controlando pela experiência e escolaridade dos indivíduos, os autores encontraram evidências que a exposição ao mercado internacional elevou os indicadores de desigualdade de renda.

Já \textcite{forbes01} estima o efeito da elevação do fluxo comercial sobre a remuneração do trabalho qualificado e não-qualificado e sobre a desigualdade salarial utilizando os dados de 36 países de 1980 até 1995. A partir de um modelo de efeitos fixos, no qual mede a desigualdade salarial pela razão entre a remuneração dos trabalhadores qualificados e não-qualificados, as evidências sugerem que o comércio internacional aprofundou o \textit{gap} salarial, reduzindo o salário dos trabalhadores sem qualificação.

Seguindo a mesma linha, \textcite{galianisanguinetti03} analisam se a liberalização comercial teve algum impacto identificável na distribuição de salários no setor manufatureiro na Argentina durante os anos noventa. Especificamente, os autores testam se os setores que experimentaram um aumento da penetração das importações foram também aqueles no qual se observou uma elevação da desigualdade salarial. Utilizando um modelo de efeitos fixos, controlando pelas características dos indivíduos e do setor, a partir dos microdados nacionais, os resultados apontam que há evidências que a liberalização contribuiu para aprofundar a desigualdade de renda via aumento do prêmio salarial por qualificação. Entretanto, a liberalização explica apenas uma proporção relativamente pequena do aumento observado.

O aumento da desigualdade observado na Argentina também foi tema para \textcite{galianiporto11} que apresentam um modelo teórico com salários rígidos abaixo do nível competitivo - por conta da presença de sindicatos e abundância dos fatores produtivos. Utilizando microdados nacionais e um modelo de efeitos fixos para estimar o log salarial dos indivíduos, as evidências apontam que a abertura comercial reduz os salários e tarifas industriais reduzem o prêmio por qualificação na indústria, havendo, por conseguinte, uma redução da desigualdade comprimindo a média salarial. Entretanto, o resultado apenas converge por considerar a distorção dos preços causada pela presença dos sindicatos.

\textcite{castilho12} estudam o efeito da globalização\footnote{Definido pelas autoras como "um termo abrangente [...] para incluir a liberalização do comércio e a integração nos mercados mundiais" \cite{castilho12}.} sobre a desigualdade de renda e pobreza a nível estadual. Utilizando um modelo de efeitos fixos e os dados da PNAD para os anos de 1987 a 2005, as autoras calculam a exposição ao comércio internacional a partir dos indicadores de fluxo comercial (penetração das importações defasada e exposição às exportações defasada) e seu efeito sobre os índices de Gini, Theil e FGT. As evidências apontam que a abertura comercial brasileira contribuiu com a elevação da pobreza e desigualdade de renda nas áreas
urbanas, havendo também uma possibilidade de relação com uma redução observada na desigualdade e pobreza nas áreas rurais.

\textcite{bayar17} se propuseram a mensurar o grau e direção da interação entre globalização, pobreza e desigualdade de renda para onze países da América Latina por meio de testes de raiz unitária em painel -- sendo sua contribuição a nível metodológico. Usando índice de Gini, FGT e corrente de comércio como porcentagem do PIB para representar, respectivamente, as medidas de desigualade de renda, pobreza e abertura comercial, os autores afirmam que o comércio internacional, no longo-prazo, reduziu os indicadores de pobreza às custas do aumento da desigualdade de renda.

Considerando experiências históricas de abertura comercial, \textcite{borrazetal12} estudam os impactos da liberalização do comércio sobre desigualdade e pobreza, focando nos preços e salários como os canais de transmissão do comércio internacional. Utilizando o modelo baseado de \textcite{dixit80} estendido por \textcite{porto06}, os autores encontraram que o Uruguai se beneficiou com a redução do nível de pobreza, apesar de haver praticamente nenhum efeito sobre a desigualdade de renda. Já o Paraguai sofreu com o aumento dos indicadores de pobreza, mesmo tendo havido uma melhoria na desigualdade de renda.

Já \textcite{porto03} examina o impacto das reformas comerciais, tanto nacionais quanto estrangeiras ocorridas na Argentina sobre o nível de pobreza. A reforma nacional seria um corte tarifário no preço dos bens de capital e de consumo importados no valor de 2\% e 8\%; já a reforma estrangeira, aumento em 7,5\% e 15,4\% nos bens agro-manufaturados e 1,6\% e 7\% nos bens manufaturados industriais.

Utilizando uma especificação que incorpora a heterogeneidade familiar em um modelo de equilíbrio parcial, o fator trabalho é desagregado em qualificado e não-qualificado com as famílias consumindo bens negociáveis e não-negociáveis. Tal qual em \textcite{borrazetal12}, há dois canais de transmissão: preços e salários. A evidência aponta que a combinação das reformas poderia causar uma redução do índice FGT entre 1,6 a 4,6 p.p., sendo majoritariamente influenciada pelos efeitos marginais da reforma doméstica.

\subsection{Equilíbrio geral} \label{subsec:eq_geral}

Considerando as experiências históricas de abertura comercial, \textcite{carneiro06} avaliam o impacto de uma expansão da abertura comercial brasileira experimentada nos anos de 1990 sobre o emprego, pobreza e desigualdade utilizando um modelo estático de equilíbrio geral computável baseado em \textcite{robinson99}, com 42 setores referente ao ano de 1996, para simular diferentes cenários de políticas. Seus resultados são levados para um modelo de microssimulação, baseado em \textcite{ganuza07}, a fim de avaliar os impactos de uma maior sobre a distribuição de renda e pobreza. A principal conclusão do artigo é que a liberalização comercial por si só não seria suficiente para reduzir a pobreza e a desigualdade no Brasil de forma significativa.

Analisando exercícios de simulação hipotética, \textcite{ferreira06} analisam os efeitos potencias da formação da ALCA sobre os níveis de pobreza e distribuição de renda brasileiros. É utilizado um modelo inter-regional estático de equilíbrio geral computável para o Brasil, calibrado a partir do SCN de 1996 com 42 setores e 52 bens. Os trabalhadores foram desagregados em dez níveis de acordo com a renda. A partir de uma integração com o modelo de microssimulação comportamental, as evidências indicaram que reduções tarifárias, por maiores que sejam, são incapazes de afetar a pobreza no Brasil.

\textcite{estrades12}, ao estimar os potenciais efeitos do acordo entre o Mercosul e a UE sobre a pobreza no Uruguai, analisa se a agenda externa do bloco é pró-pobre. Utilizou-se um modelo global de equilíbrio geral, MIRAGE, calibrado para o ano de 2004, contendo 19 países e 30 setores, integrado a uma abordagem de microssimulação para analisar a pobreza. Seus resultados indicaram que o acordo teria um impacto significativo nos fluxos comerciais entre os dois blocos. Os países-membros do Mercosul aumentariam as exportações agrícolas para a UE e as importações industriais da UE. O bem-estar aumenta em todos os países participantes do acordo, mas é mais pronunciado para os dois pequenos países do Mercosul: Paraguai e Uruguai. Neste, o bem-estar aumenta para diferentes classes de famílias, mas as mais ricas são as que mais se beneficiam. Apesar disso, a desigualdade diminui com o acordo e os índices de pobreza diminuem em todo o país.

\textcite{corong14} estuda o impacto das reduções tarifárias nas Filipinas sobre a desigualdade de renda e pobreza entre homens e mulheres. Utilizou-se um modelo dinâmico nacional da tradição PHILGEM, com sete tipos de trabalho, 105 setores e dois gêneros para o ano 2000. Como resultado, o autor afirma que a redução tarifária beneficiou mais as mulheres (\textit{pro-women}) em função da expansão de setores cuja participação feminina era maior e registrou queda dos indicadores de pobreza e elevação dos indicadores de desigualdade de renda. 


