
% ----------------------------------------------------------
% CAPÍTULO 02 - REVISÃO DE LITERATURA
% ----------------------------------------------------------

\chapter{Revisão de literatura}\label{cha:revisao_de_literatura}

Este capítulo contextualiza a contribuição, tanto teórica quanto empírica, da literatura econômica para a temática desta dissertação. A primeira seção se concentra nos canais de transmissão que associam o comércio internacional à desigualdade de renda e pobreza, discutindo a natureza desse vínculo e o comportamento esperado. A segunda e última seção apresenta as evidências empíricas existentes sobre os canais de transmissão debatidos na seção anterior, no qual pode ser percebido a dificuldade em criar uma convergência acerca dos efeitos do comércio internacional sobre os indicadores de desigualdade de renda e pobreza.



\section{Os canais de transmissão}\label{sec:canais_de_transmissao}

De início, é importante afirmar que estabelecer uma associação entre comércio internacional e indicadores de desigualdade de renda e pobreza é uma atividade desafiadora. A própria mensuração de desigualdade e pobreza é bastante complexa, sendo, por si só, tema exclusivo de diversos estudos \cite{neri06, soares09, hoffmann19}. Ademais, há outro grande desafio em desvincilhar os próprios canais de transmissão entre si, uma vez que são interdependentes e sujeitos a influência de outros tipos de políticas e eventos econômicos \cite{bannisterthugge01}.

Pode-se entender o comércio internacional enquanto um choque exógeno sobre os preços relativos de uma economia com fortes efeitos distributivos entre e dentro dos países\footnote{O comércio internacional pode afetar os preços relativos de uma economia de diversas formas: por variações cambiais, inovação, concorrência, etc. No modelo H-O, a abertura comercial promove a convergência dos preços relativos entre os dois países a partir da entrada do bem por um preço abaixo do praticado em autarquia.}. O resultado final desse choque depende da estrutura de mercado e particularmente do setor de distribuição\footnote{Pode ser entendido como uma categoria genérica para agrupar todos os canais de transmissão entre o comércio internacional e a desigualdade de renda e pobreza existentes em uma economia \cite{winters02}} \cite{winters02}. Desse modo, frente a um processo de liberalização comercial, se a estrutura de mercado e setor de distribuição sao aquém do esperado, é possível que haja determinadas regiões do país e setores produtivos que fiquem isolados dos efeitos positivos desse processo \cite{bannisterthugge01}.

Intuitivamente, o primeiro canal de transmissão que se pode imaginar é o crescimento econômico, uma vez que um maior crescimento indica maiores oportunidades de emprego e de geração de renda, que afetam diretamente a pobreza e desigualdade\footnote{Há uma extensa literatura econômica acerca dos efeitos do crescimento pró-pobre e pró-rico (\textit{pro-poor and pro-rich growth}) sobre a desigualdade de renda. Sobre isso, pode-se citar \textcite{ravallion04, thorbecke22}.}. Entretanto, sua conexão com o comércio internacional é indefinida, sendo também, por si só, tema de diversos estudos que buscam investigar essa conexão \cite{anderson16, dix17, gnangnon18}.

A primeira subseção discute os canais de transmissão conhecidos na literatura entre o comércio internacional e a desigualdade de renda. Já a segunda subseção discute os canais de transmissão que afetam a pobreza. O Quadro~\ref{quad:canais_de_transmissao} resume em tópicos a discussão realizada a seguir.


\subsection{Desigualdade de renda} \label{subsec:desigualdade_de_renda}

O prêmio salarial por qualificação é um canal através do qual o comércio internacional pode afetar a desigualdade de renda, sendo entendido enquanto a diferenciação nos salários devido ao nível de qualificação, habilidades ou educação de um trabalhador. Sua variação pode ocorrer a partir de: 1- um aumento nos retornos de determinadas ocupações associadas a um nível de escolaridade maior; 2- um deslocamento da produção de bens intermediários intensivos em qualificação dos países desenvolvidos para os em desenvolvimento; 3- uma mudança tecnológica com viés para maior qualificação; e 4- uma alteração da composição nos produtos produzidos dentro das indústrias em prol dos intensivos em trabalho qualificado \cite{goldbergpavcnik04}.

Seu comportamento é positivamente relacionado com os indicadores de desigualdade. Ou seja, uma redução no prêmio salarial por qualificação diminui a desigualdade de renda. Isso ocorre porque o prêmio salarial afeta diretamente o \textit{gap} entre dois ou vários salários, agravando a desigualdade de renda. Esse resultado é convergente com os modelos teóricos de comércio internacional, em especial com o modelo H-O\footnote{Considerando dois países, desenvolvido e em desenvolvimento, cujo primeiro país é abundante em trabalho qualificado e o segundo, em não-qualificado, o teorema SS garante que haveria uma redução salarial dos trabalhadores qualificados, aumentando o dos não-qualificados quando ambos se engajam em comércio.}. A Figura~\ref{fig:modelo_h-o} ilustra essa dinâmica.

\begin{figure}[h]
	\centering
	\includegraphics[width = 12cm]{Imagens/001.ai}
	\caption{Resultado do comércio internacional no modelo H-O}
	\label{fig:modelo_h-o}
	\footnotesize
	Fonte: elaboração própria (2024)
\end{figure}

O prêmio salarial por setor\footnote{Geralmente o prêmio salarial por setor se refere a parte do salário não explicada pelas características observáveis dos trabalhadores.} é um outro possível canal de transmissão, entendido enquanto a variação salarial que pode existir entre trabalhadores empregados em setores diferentes da atividade econômica. Essa diferenciação ocorre através do corte de tarifas - no curto e médio-prazo no qual não há livre mobilidade do fator trabalho entre setores - e através de mudanças na produtividade setorial.

Tal qual o anterior, seu comportamento é positivamente relacionado com a desigualdade. A redução no prêmio por setor reduz a desigualdade de renda; mas é necessário cumprir um condicionante: o corte tarifário precisa ocorrer no setor intensivo em trabalho qualificado e com menor prêmio salarial \cite{goldbergpavcnik04}. Caso contrário, a redução no prêmio provoca o aumento da desigualdade de renda. Esse comportamento é convergente com o modelo H-O, uma vez que o produto com maior corte tarifário é o produto importado, cujo preço relativo será reduzido, como garante o teorema SS.

A informalidade também é um canal de transmissão capaz de influir sobre os indicadores de desigualdade de renda. Seu comportamento também é positivamente relacionado: a abertura comercial pode aumentar a informalidade, causando o aumento da desigualdade de renda. O argumento seria que a competição estrangeira geraria incentivos para as firmas domésticas a cortarem custos, optando por trabalhadores informais \cite{goldbergpavcnik04} dado que não há a necessidade de cumprir leis ou regulações trabalhistas, além da média salarial ser substancialmente menor \cite{bargain14}.

O mercado de crédito pode ser influenciado pelo comércio internacional, sendo, também, um canal de transmissão sobre os indicadores de desigualdadade. Sua eficiência é negativamente relacionada com os indicadores de desigualdade. \textcite{banerjee04} desenvolvem um modelo dinâmico de comércio internacional com fatores específicos que incorpora imperfeição na mobilidade do capital e trabalho\footnote{No modelo, só há a transição de trabalho para capital se o indivíduo tiver uma quantidade mínima de capital humano.}. Frente a abertura comercial, observa-se, no modelo, o aumento da taxa de juros nos países com mercado de crédito mais robusto e consequente aumento da produtividade média, ao passo que os países com mercado de crédito pouco eficiente sofrem uma redução em sua taxa de juros, fazendo com que o capital permaneça nos setores menos produtivos - uma vez que aumenta a probabilidade de inadimplência. O efeito sobre a distribuição de renda é um \textit{trade-off} para os países em desenvolvimento: maior crescimento econômico no longo prazo às custas de alta volatilidade no curto-prazo.

O último canal de transmissão é o que a literatura econômica chama de bens não-negociáveis. Entende-se por isso todos os bens cuja características dificultam ou impedem de serem exportados ou importados. De acordo com \textcite{xu03}\footnote{Utiliza-se um modelo H-O com dois países (norte e sul), dois fatores (trabalho qualificado e não-qualificado) e quatro bens (1, 2, 3 e 4 numa escala crescente de complexidade.}, os efeitos da abertura comercial sobre a desigualdade de renda nos países em desenvolvimento são em forma de U. Para níveis de proteção comercial inicialmente elevados, a abertura reduz a desigualdade, enquanto para níveis de proteção mais baixos, a abertura aumenta a desigualdade.

Isso ocorre por causa da gama de bens não-negociáveis: a proteção comercial transforma alguns bens negociáveis em não-negociáveis. Se esses bens são intensivos em trabalho qualificado, uma progressiva abertura comercial, via redução tarifária, aumenta os incentivos para importar. A consequência disso é a redução do salário do trabalho qualificado no país, reduzindo o \textit{gap} salarial - como visto no Teorema SS. Entretanto, a redução tarifária também provoca a piora dos termos de troca, o que torna por incentivar as exportações do país - o que elevaria o salário do trabalho qualificado, aumentando a desigualdade. Uma elevada abertura comercial faria com que o "efeito exportação" superasse o "efeito importação", criando, por conseguinte, a forma de U para os países em desenvolvimento.


\subsection{Pobreza} \label{subsec:pobreza}

Como discutido anteriormente, uma aberta comercial afeta tanto os preços relativos de uma economia quanto pode transformar bens não-negociáveis em negociáveis. Essas alterações podem ser elencadas como um canal de transmissão do comércio internacional sobre os indicadores de pobreza. Os indivíduos mais pobres podem ser beneficiados por uma abertura comercial, a partir da redução do preço dos bens importados, como alimentos básicos e farmacêuticos, gerando o aumento da renda real \cite{bannisterthugge01}. Como geralmente os indivíduos mais pobres são produtores líquidos de bens voltados para a exportação (como na Agricultura, por exemplo), o aumento da demanda por exportações pode elevar o preço dos produtos e, por conseguinte, estimular o aumento da produção, emprego e renda do setor \cite{bannisterthugge01}.

A alteração no preço relativo dos fatores de produção também é um canal de transmissão sobre os indicadores de pobreza, uma vez que afeta diretamente a remuneração do trabalho qualificado e não-qualificado. Como já visto no modelo H-O, o comércio internacional pode beneficiar os mais pobres caso a abertura se reverta numa elevação da produtividade marginal do trabalho não-qualificado - como demonstrado na Figura~\ref{fig:modelo_h-o} \cite{bannisterthugge01}.

O comércio internacional também pode afetar as receitas e capacidade de gastos do governo, uma vez que políticas de liberalização comercial geralmente reduzem as receitas tarifárias. O canal de transmissão sobre a pobreza se baseia no argumento que essa redução na receita tarifária teria que ser compensada pelo governo através de cortes em programas sociais ou criação de novos impostos, afetando desproporcionalmente os mais pobres \cite{bannisterthugge01}. Entretanto, o resultado é ambíguo, dependendo do cenário: por exemplo, se as tarifas forem inicialmente altas, o corte tarifário elevará o fluxo comercial de tal maneira a compensar as perdas tributárias, além de diminuir os incentivos ao contrabando e corrupção \cite{bannisterthugge01}.

A literatura econômica associa reformas comerciais a maiores fluxos de investimento externo com \textit{spillovers} sobre tecnologias, novas práticas de negócios e outros efeitos sobre as empresas nacionais que aumentam o nível geral de produtividade\footnote{Reformas comerciais também resultam na melhoria da formação de capital humano, que gera um significante efeito sobre inovação \cite{bannisterthugge01}.} \cite{bannisterthugge01}. Entretanto, é possível que esse tipo de aumento de produtividade engendre um crescimento econômico que beneficie desproporcionalmente os mais ricos, o que agravaria os indicadores de pobreza.

Por fim, aberturas comerciais podem facilitar a diversificação da pauta exportadora de um país, tornando-o menos dependente de um único mercado ou bem; como pode, também, tornar a economia mais vulnerável a choques externos. O canal de transmissão com indicadores de pobreza, nesse caso, se dá quando esse choque recai sobre setores intensivos em trabalho não-qualificado, como Agricultura ou qualquer setor muito caracterizado por trabalho informal, afetando desproporcionalmente os mais pobres \cite{bannisterthugge01}. Desse modo, a abertura comercial gera efeitos ambíguos: há evidências de elevação do crescimento econômico, entretanto, com altos níveis de volatilidade macroeconômica - o que, mais uma vez, afeta desproporcionalmente os mais pobres.

\begin{quadro}[h]
	\begin{threeparttable}
		\centering
		\caption{Canais de transmissão entre comércio internacional e a desigualdade de renda e pobreza}
		\footnotesize
		\label{quad:canais_de_transmissao}
		\begin{tabular}{|| m{7.5cm} | m{7.5cm} ||}
			\hline \hline
			\multicolumn{1}{||c|}{\textbf{desigualdade de renda}} & \multicolumn{1}{c||}{\textbf{pobreza}} \\ \hline
			\begin{itemize}
				\item prêmio salarial por qualificação
				\item prêmio salarial por setor
				\item emprego informal
				\item imperfeições do mercado de crédito
				\item quantidade de bens não-negociáveis
			\end{itemize} &
			\vspace{0.2cm}
			alteração no:
			\begin{itemize}
			\item preço e acesso dos produtos negociáveis
			\item preço relativo dos fatores de produção
			\item receitas e capacidade de gastos do governo
			\item incentivos de investimentos e inovação
			\item vulnerabilidade à choques externos
			\end{itemize} \\ \hline \hline
		\end{tabular}
		\begin{tablenotes}
			\scriptsize
			\item Fonte: \textcite{bannisterthugge01, xu03, goldbergpavcnik04, banerjee04}.
		\end{tablenotes}
	\end{threeparttable}
\end{quadro}



\section{As evidências empíricas} \label{sec:evidencias_empiricas}

Conforme discutido no capítulo~\ref{cha:introducao}, apesar da teoria econômica convergir para a noção que o comércio internacional é um fator capaz de reduzir a desigualdade de renda e pobreza, as evidências empíricas apontam para distintos cenários. Uma possível justificativa seja a ausência de uma resposta única para a questão. Os efeitos do comércio internacional sobre a desigualdade de renda e pobreza podem ser dependentes da estrutura produtiva do país, bem como da composição de sua pauta exportadora e da distribuição funcional da renda. Esses fatores, em conjunto, produzem efeitos heterogêneos quando são expostos a políticas comerciais liberalizantes.

\textcite{winters02} elenca cinco razões para o \textit{mismatch} entre a teoria e as evidências empíricas: 1- distribuição funcional da renda; 2- dimensionalidade dos modelos; 3- pressuposto de mobilidade dos fatores; 4- equilíbrio diverso; e 5- comportamento do preço dos bens não negociáveis para \textit{market-clearing}.

Optou-se por conduzir a discussão tendo a metodologia como fio condutor; desse modo, torna-se mais evidente as contribuições e limitações de cada método ao abordar a temática em questão, bem como seus pontos de concordância e discordância.


\subsection{Equilíbrio parcial} \label{subsec:eq_parcial}

Um dos mais tradicionais canais de transmissão, no equilíbrio parcial, é o mercado de trabalho. Nesse caminho, \textcite{borjas94} buscam verificar se a tendência de exposição dos setores altamente concentrados ao comércio internacional durante 1963 a 1988 nos Estados Unidos podem ser responsáveis por grande parte das tendências na desigualdade salarial. A partir do modelo de séries temporais, controlando pela experiência e escolaridade dos indivíduos, os autores encontram que essa exposição tornou por aumentar a desigualdade de renda.

Já \textcite{forbes01} estima o efeito da elevação do fluxo comercial sobre a remuneração do trabalho qualificado e não-qualificado e sobre a desigualdade salarial utilizando os dados de 36 países de 1980 até 1995. A partir de um modelo de efeitos fixos, no qual mede a desigualdade salarial pela razão entre a remuneração dos trabalhadores qualificados e não-qualificados, as evidências sugerem que o comércio internacional aprofundou o \textit{gap} salarial, reduzindo o salário dos trabalhadores sem qualificação.

Seguindo a mesma linha, \textcite{galianisanguinetti03} analisam se a liberalização comercial teve algum impacto identificável na distribuição de salários no setor manufatureiro na Argentina durante os anos noventa. Especificamente, os autores testam se os setores que experimentaram um aumento da penetração das importações foram também aqueles no qual se observou uma elevação da desigualdade salarial. Utilizando um modelo de efeitos fixos, controlando pelas características dos indivíduos e do setor, a partir dos microdados nacionais, os resultados apontam que há evidências que a liberalização contribuiu para aprofundar a desigualdade de renda via aumento do prêmio salarial por qualificação. Entretanto, a liberalização explica apenas uma proporção relativamente pequena do aumento observado.

O aumento da desigualdade observado na Argentina também foi tema para \textcite{galianiporto11} que apresentam um modelo teórico com salários rígidos abaixo do nível competitivo - por conta da presença de sindicatos e abundância dos fatores produtivos. Utilizando microdados nacionais e um modelo de efeitos fixos para estimar o log salarial dos indivíduos, as evidências apontam que a abertura comercial reduz os salários e tarifas industriais reduzem o prêmio por qualificação na indústria, havendo, por conseguinte, uma redução da desigualdade comprimindo a média salarial. Entretanto, o resultado apenas converge por considerar a distorção dos preços causada pela presença dos sindicatos.

Considerando experiências históricas de abertura comercial, \textcite{castilho12} estudam o efeito da globalização\footnote{Definido pelas autoras como "um termo abrangente [...] para incluir a liberalização do comércio e a integração nos mercados mundiais" \cite{castilho12}.} sobre a desigualdade de renda e pobreza a nível estadual. Utilizando um modelo de efeitos fixos e os dados da PNAD para os anos de 1987 a 2005, as autoras calculam a exposição ao comércio internacional a partir dos indicadores de fluxo comercial (penetração das importações defasada e exposição às exportações defasada) e seu efeito sobre os índices de Gini, Theil e FGT. As evidências apontam que a abertura comercial brasileira contribuiu com a elevação da pobreza e desigualdade de renda nas áreas
urbanas, havendo também uma possibilidade de relação com uma redução observada na desigualdade e pobreza nas áreas rurais.

\textcite{bayar17} se propuseram a mensurar o grau e direção da interação entre globalização, pobreza e desigualdade de renda para onze países da América Latina, empregando \textit{second generation panel unit root tests}, sendo sua contribuição a nível metodológico. Usando índice de Gini, FGT e corrente de comércio como porcentagem do PIB para representar, respectivamente, as medidas de desigualade de renda, pobreza e abertura comercial, os autores afirmam que o comércio internacional, no longo-prazo, reduziu os indicadores de pobreza às custas do aumento da desigualdade de renda.

Partindo para analisar os estudos de caso, \textcite{borrazetal12} estudam os impactos da liberalização do comércio sobre desigualdade e pobreza, focando nos preços e salários como os canais de transmissão do comércio internacional. Utilizando o modelo baseado de \textcite{dixit80} estendido por \textcite{porto06}, os autores encontraram que o Uruguai se beneficiou com a redução do nível de pobreza, apesar de haver praticamente nenhum efeito sobre a desigualdade de renda. Já o Paraguai sofreu com o aumento dos indicadores de pobreza, mesmo tendo havido uma melhoria na desigualdade de renda.

\subsection{Equilíbrio geral} \label{subsec:eq_geral}

Considerando as experiências históricas de abertura comercial, \textcite{porto03} examina o impacto das reformas comerciais, tanto nacionais quanto estrangeiras\footnote{A reforma nacional seria o corte tarifário; a estrangeira seria a eliminação de subsídios e barreiras tarifárias e não-tarifárias}, ocorridas na Argentina sobre o nível de pobreza. Utilizando \textit{framework} que incorpora a heterogeneidade familiar em um modelo de equilíbrio geral, o trabalho é desagregado em qualificado e não-qualificado e as famílias consomem bens negociáveis e não-negociáveis. Tal qual em \textcite{borrazetal12}, há dois canais de transmissão: preços e salários. A evidência aponta que a combinação das reformas poderia causa uma redução do índice FGT entre 1,6 a 4,6 p.p., sendo majoritariamente influenciado pelos efeitos marginais da reforma doméstica.

\textcite{carneiro06} avaliam o impacto de uma expansão da abertura comercial brasileira experimentada nos anos de 1990 sobre o emprego, pobreza e desigualdade utilizando um modelo de equilíbrio geral computável baseado em \textcite{robinson99} para simular diferentes cenários de políticas\footnote{São oito simulações, a saber: 1- $\uparrow$ 10\% produtividade; 2- $\uparrow$ 10\% preço das importações; 3 e 4- efeitos de uma mudança de 50\% nas tarifas de importação; 5- $\uparrow$ 10\% da taxa de subsídios às exportações; 6- avaliar os impactos da ALCA; 7- avaliar os impactos da OMC; e 8- $\downarrow$ 10\% câmbio.}, usando os resultados dessa abordagem para criar microssimulações contrafactuais, baseadas em \textcite{ganuza07}, e avaliar os impactos de uma maior abertura comercial na distribuição de renda familiar e nas taxas de pobreza, sendo 1996 o ano-base. A principal conclusão, a partir das evidências, é que a liberalização comercial por si só não é suficiente para reduzir a pobreza e a desigualdade no Brasil de forma significativa.

Considerando os estudos de caso, \textcite{ferreira06}, através de um modelo inter-regional estático de equilíbrio geral para o Brasil\footnote{O referido modelo foi calibrado com os dados da SCN de 1996, contendo 42 setores, 52 produtos e todas as 27 Unidades Federativas. Os trabalhadores foram desagregados em dez níveis de acordo com a renda}, tendo 2001 como o ano-base, integrado a um modelo de microssimulação - calibrado com os dados da PNAD 2001, analisa os efeitos potencias da formação da ALCA sobre os níveis de pobreza e distribuição de renda brasileiros. A evidência aponta para o fato que mesmo mudanças tarifárias grandes como as aqui simuladas não trariam um forte impacto sobre
a pobreza no Brasil, embora os resultados estejam concentrados nos domicílios mais pobres.

\textcite{estrades12}, ao estimar os potenciais efeitos do acordo entre o Mercosul e a UE sobre a pobreza no Uruguai, analisa se a agenda externa do bloco é pró-pobre. Utilizando o modelo de equilíbrio geral, MIRAGE\footnote{\textit{Modelling International Relationships in Applied General Equilibrium} é um modelo de equilíbrio geral computável multissetorial e multirregional dedicado à análise de políticas comerciais, calibrado a partir dos dados do GTAP para o ano de 2004, contendo um total de 19 países, contemplando os quatro membros do Mercosul e os 27 países da UE (agregados), além de 30 setores produtivos.}, integrado a uma abordagem micro para analisar a pobreza, os resultados indicam que  o acordo teria um impacto significativo nos fluxos comerciais entre os dois blocos. Os países-membros do Mercosul aumentariam as exportações agrícolas para a UE e as importações industriais da UE. O bem-estar\footnote{Obtido no modelo a partir da desagregação do agente representativo em público e privado para todos os países do modelo, subdividindo o agente privado representativo de países específicos em um número diversificado de domicílios.} aumenta em todos os países participantes do acordo, mas é mais pronunciado para os dois pequenos países do Mercosul: Paraguai e Uruguai. Neste, o bem-estar aumenta para diferentes classes de famílias, mas as mais ricas são as que mais se beneficiam. Apesar disso, a desigualdade diminui com o acordo e os índices de pobreza diminuem em todo o país.

\textcite{campostimini22} estudam o impacto da criação do Mercosul sobre o bem-estar dos países-membros: Brasil, Argentina, Paraguai e Uruguai. Utilizando um modelo de equação gravitacional moderno de equilíbrio geral baseado em \textcite{arkolakis21}, os autores afirmam que o país-membro mais beneficiado pelo bloco regional foi Argentina, ao passo em que o Brasil foi o menos beneficiado. Entretanto, após testes de cenários contrafactuais, os autores declaram que uma eventual saída do bloco seria prejudicial ao país por conta dos custos políticos e aumento de incerteza.


