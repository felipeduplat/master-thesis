
%---------------------------------------------------------------------
% RESUMO
%---------------------------------------------------------------------

% PT-BR:
\begin{resumo}
	\SingleSpacing
	
	Apesar da maioria dos modelos teóricos de economia internacional convergirem para a compreensão de que o comércio pode ser um fator positivo para o desenvolvimento econômico de um país, com efeitos positivos sobre os indicadores de desigualdade de renda e pobreza, as evidências empíricas apontam para distintos cenários. Uma das possíveis explicações para essas divergências está relacionada a estrutura produtiva de cada país e período, assim como à relação entre a estrutura produtiva e a distribuição de renda. Nesse sentido, visando compreender melhor a questão, a presente dissertação tem como objetivo estimar os efeitos de uma maior abertura comercial sobre a distribuição da renda familiar e sobre os índices de pobreza no Brasil. Para isso, utiliza-se um modelo nacional de equilíbrio geral computável integrado a um modelo de microssimulação. Enquanto o primeiro modelo permite calcular efeitos agregados e setoriais da redução de barreiras tarifárias, o segundo modelo permite acessar os resultados a nível individual, calculando os possíveis efeitos sobre indicadores de pobreza e desigualdade. Os resultados indicaram que o comércio internacional exerce pouca influência sobre a desigualdade de renda e pobreza. As variações registradas foram bastante modestas, sobretudo se tratando do efeito sobre a pobreza extrema e sobre a desigualdade de renda, sendo praticamente nulo. Mesmo assim, observou-se uma modesta redução da desigualdade de renda e pequeno aumento da pobreza absoluta e extrema. Uma possível razão que explique esse resultado esteja no fato que as barreiras tarifárias computadas no modelo já não são altas o suficiente para que uma nova redução tarifária consiga impor efeitos expressivos sobre os indicadores observados.
	
	\noindent 
	\textbf{Palavras-chave}: Comércio internacional. Desigualdade de renda. Pobreza. Equilíbrio Geral Computável. Microssimulação comportamental. \\
	\textbf{Classificação JEL}: F10, F14, I32.
\end{resumo}


