
%---------------------------------------------------------------------
% RESUMO
%---------------------------------------------------------------------

% PT-BR:
\begin{resumo}
	\SingleSpacing
	
	Apesar dos modelos teóricos de economia internacional convergirem para a compreensão de que o comércio pode ser um fator positivo para o desenvolvimento econômico de um país, reduzindo os índices de desigualdade de renda e pobreza, as evidências empíricas, até então, demonstram resultados dúbios, não havendo nenhum tipo de convergência. Entretanto, isso não necessariamente significa que não haja uma resposta conclusiva na literatura, mas sim pode indicar que não existe uma resposta única. Dado esse cenário, a presente dissertação tem como objetivo estimar os efeitos de uma maior abertura comercial sobre a distribuição da renda familiar e sobre os índices de pobreza no Brasil através do modelo nacional de equilíbrio geral para simular diferentes cenários de políticas de liberalização comercial integrado a uma abordagem de microssimulações contrafactuais para capturar as respostas comportamentais dos indivíduos. 
	
	\noindent 
	\textbf{Palavras-chaves}: Comércio internacional. Desigualdade de renda. Pobreza. Equilíbrio Geral Computável. Microssimulação.
\end{resumo}


