
%---------------------------------------------------------------------
% RESUMO
%---------------------------------------------------------------------

% PT-BR:
\begin{resumo}
	\SingleSpacing
	
	Apesar dos modelos teóricos de economia internacional convergirem para a compreensão de que o comércio pode ser um fator positivo para o desenvolvimento econômico de um país com \textit{spillovers} positivos sobre seus indicadores de desigualdade de renda e pobreza, as evidências empíricas apontam para distintos cenários, sem haver consenso na literatura econômica sobre seus efeitos. Isso não indica, necessariamente, que os resultados sejam inconclusivos, mas sim que não existe uma resposta única para a questão. Dado esse cenário, a presente dissertação tem como objetivo estimar os efeitos de uma maior abertura comercial sobre a distribuição da renda familiar e sobre os índices de pobreza no Brasil. Para isso, utiliza-se um modelo nacional de equilíbrio geral para simular os efeitos de curto-prazo de uma redução tarifária em 10\% integrado a uma abordagem de microssimulações contrafactuais para capturar as respostas comportamentais dos indivíduos. Os resultados indicaram que o comércio internacional exerce pouca influência sobre os indicadores de desigualdade de renda e pobreza. As variações registradas foram bastante modestas, sobretudo tratando-se do efeito sobre a pobreza extrema e sobre a desigualdade de renda, sendo praticamente nulo. Uma possível razão que explique esse resultado esteja no fato que as barreiras tarifárias brasileiras já não são altas o suficiente para que uma nova redução tarifária consiga impor efeitos expressivos sobre os indicadores observados.
	
	\noindent 
	\textbf{Palavras-chaves}: Comércio internacional. Desigualdade de renda. Pobreza. Equilíbrio Geral Computável. Microssimulação comportamental.
\end{resumo}


