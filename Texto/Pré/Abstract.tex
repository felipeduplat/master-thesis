
%---------------------------------------------------------------------
% ABSTRACT
%---------------------------------------------------------------------

% ENG:
\begin{resumo}[Abstract]
	\begin{otherlanguage*}{english}
		\SingleSpacing
		
		Despite most theoretical models of international economics converging on the understanding that trade can be a positive factor for the economic development of a country, with positive effects on income inequality and poverty indicators, empirical evidence points to distinct scenarios. This does not necessarily indicate that the results are inconclusive, but rather that there is no single answer to the question. Given this scenario, this dissertation aims to estimate the effects of greater trade openness on the distribution of family income and poverty rates in Brazil. For this, a national computable general equilibrium model integrated with a microsimulation model is used. While the first model allows calculating aggregate and sectoral effects of reducing tariff barriers, the second model allows accessing results at the individual level, calculating the possible effects on poverty and inequality indicators. The results indicated that international trade has little influence on income inequality and poverty. The variations recorded were quite modest, especially when it comes to the effect on extreme poverty and income inequality, which was practically negligible. Even so, there was a modest reduction in income inequality and a small increase in absolute and extreme poverty. A possible reason that explains this result is the fact that Brazilian tariff barriers are no longer high enough for a new tariff reduction to be able to impose significant effects on the observed indicators.
		
		\noindent 
		\textbf{Keywords}: International trade. Wage inequality. Poverty. Computable General Equilibrium. Behavioral microsimulation. \\
		\textbf{JEL Classification}: F10, F14, I32.
	\end{otherlanguage*}
\end{resumo}


