
%---------------------------------------------------------------------
% ABSTRACT
%---------------------------------------------------------------------

% ENG:
\begin{resumo}[Abstract]
	\begin{otherlanguage*}{english}
		\SingleSpacing
		
		Despite theoretical models of international economics converging on the understanding that trade can be a positive factor for the economic development of a country, with positive spillovers on its indicators of income inequality and poverty, empirical evidence points to distinct scenarios, without a consensus in the economic literature about their effects. This does not necessarily indicate that the results are inconclusive, but rather that there is no single answer to the question. Given this scenario, the present dissertation aims to estimate the effects of greater trade openness on the distribution of family income and on poverty rates in Brazil through a national general equilibrium model, simulating the short-term effects of a tariff reduction by 10\%, integrated with a counterfactual microsimulations approach to capture individuals' behavioral responses. The results indicated that international trade has little influence on indicators of income inequality and poverty. The recorded variations were quite modest, especially when it comes to the effect on extreme poverty and income inequality, which was practically negligible. A possible reason that explains this result is the fact that Brazilian tariff barriers are no longer high enough for a new tariff reduction to impose significant effects on the observed indicators.
		
		\noindent 
		\textbf{Keywords}: International trade. Wage inequality. Poverty. Computable General Equilibrium. Behavioral microsimulation.
	\end{otherlanguage*}
\end{resumo}


