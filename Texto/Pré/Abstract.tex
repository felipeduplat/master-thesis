
%---------------------------------------------------------------------
% ABSTRACT
%---------------------------------------------------------------------

% ENG:
\begin{resumo}[Abstract]
	\begin{otherlanguage*}{english}
		\SingleSpacing
		
		Despite the theoretical models of international economics converging on the understanding that trade can be a positive factor for the economic development of a country, reducing income inequality and poverty indices, empirical evidence, so far, has shown dubious results, not there is no kind of convergence. However, this does not necessarily mean that there is no conclusive answer in the literature, but it may indicate that there is no single answer. Given this scenario, the present dissertation aims to estimate the effects of greater trade liberalization on the distribution of family income and on poverty rates in Brazil through the national general equilibrium model to simulate different scenarios of trade liberalization policies integrated with a counterfactual microsimulation approach to capture individuals' behavioral responses.
		
		\noindent 
		\textbf{Keywords}: International trade. Wage inequality. Poverty. Computable General Equilibrium.
	\end{otherlanguage*}
\end{resumo}


