
%---------------------------------------------------------------------
% FICHA CATALOGRÁFICA
%---------------------------------------------------------------------

% São duas opções possíveis: usar um PDF pronto ou usar o comando abaixo. Se quiser usar o PDF, basta inserir um arquivo .pdf na pasta "Pré" chamado "Ficha catalográfica". Caso queira usar o comando, basta editá-lo abaixo de acordo com suas necessidades e depois remover o .pdf da pasta "Pré".

\newcommand{\PalavraschaveTexto}{Comércio internacional. Desigualdade de renda. Pobreza. Equilíbrio Geral Computável. Microssimulação.}

\newcommand{\insereFichaCatalografica}{ 
		\IfFileExists{Pré/Ficha catalográfica.pdf}
		{\includepdf[pages=-]{Pré/Ficha catalográfica.pdf}}
		{
			\begin{fichacatalografica}%\color{blue}
			\vspace*{\fill}					% Posição vertical
			\hrule							% Linha horizontal
			\begin{center}					% Minipage Centralizado
				\begin{minipage}[c]{12.5cm}		% Largura
					
					Luz, Felipe Duplat
					
					\hspace{0.5cm} \imprimirtitulo  / \imprimirautor. --
					\imprimirlocal, \imprimirdata.
					
					\hspace{0.5cm} \pageref{LastPage} p. : il. color \\
					
					\hspace{0.5cm} \imprimirtipotrabalho (mestrado) - \imprimirinstituicao. \ImprimirProgramaPos, do \ImprimirSetor. \\ 
					
					\hspace{0.5cm} Orientador: \imprimirorientador \\
					
					\hspace{0.5cm} Coorientadora: \imprimircoorientadora \\
					
					\hspace{0.5cm} Defesa: \imprimirlocal, \imprimirdata. \\
					
					\begin{minipage}{.96\textwidth}
						\hspace{0.5cm} 1. \PalavraschaveTexto. I. \imprimirinstituicao. \ImprimirSetor. \ImprimirProgramaPos. II. \imprimirorientador. III. \imprimircoorientadora. IV. \imprimirtitulo\\ 
						
						\hspace{92mm} CDU \imprimircdu \\
					\end{minipage}
				\end{minipage}
			\end{center}
			\hrule
		\end{fichacatalografica}
	}
}

\insereFichaCatalografica
\cleardoublepage


