
% ----------------------------------------------------------
% DADOS
% ----------------------------------------------------------

% Tipo de TCC (Monografia, Dissertação, Tese ou Relatório Técnico):
\tipotrabalho{Dissertação}

% Informações do TCC:
\titulo{Comércio internacional, desigualdade de renda e pobreza: uma análise integrada de equilíbrio geral e microssimulação para o Brasil} % título.
\autor{Felipe Duplat Luz}                % autor.
\local{Curitiba}                         % cidade.
\data{2024}                              % ano.
\orientador{Vinícius de Almeida Vale}    % orientador  - se homem.
%\orientadora{}                          % orientadora - se mulher.
%\coorientador{}                         % co-orientador  - se homem
\coorientadora{Kênia Barreiro de Souza}  % co-orientadora - se mulher
%\scoorientador{}                        % segundo co-orientador  - se mulher
%\scoorientadora{}                       % segunda co-orientadora - se mulher

% Adicionar referências:
\addbibresource{Arquivos/Referências.bib}

% Cabeçalho da capa:
\instituicao{Universidade Federal do Paraná}
\def \ImprimirSetor{Setor de Ciências Sociais Aplicadas}
\def \ImprimirProgramaPos{Programa de Pós-Graduação em Desenvolvimento Econômico}
\def \ImprimirCurso{}
\preambulo{Dissertação de mestrado apresentada ao Programa de Pós-Graduação em Desenvolvimento Econômico do Setor de Ciências Sociais Aplicadas da Universidade Federal do Paraná como requisito parcial para obtenção do título de mestre em Desenvolvimento Econômico}

% Informações complementares:
\newcommand{\imprimirCurso}{}
\newcommand{\imprimirDataDefesa}{26 de fevereiro de 2024}
\newcommand{\imprimircdu}{02:141:005.7}

% Comandos de dados - Data da apresentação
\providecommand{\imprimirdataapresentacaoRotulo}{}
\providecommand{\imprimirdataapresentacao}{}
\newcommand{\dataapresentacao}[2][\dataapresentacaoname]{\renewcommand{\dataapresentacao}{#2}}

% Comandos de dados - Nome do Curso
\providecommand{\imprimirnomedocursoRotulo}{}
\providecommand{\imprimirnomedocurso}{}
\newcommand{\nomedocurso}[2][\nomedocursoname]
  {\renewcommand{\imprimirnomedocursoRotulo}{#1}
\renewcommand{\imprimirnomedocurso}{#2}}


